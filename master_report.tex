%    Template for seminar reports
% Seminar Current Topics in Computer Vision and Machine Learning
% Summer Semester 2015
% Computer Vision Group, Visual Computing Institute, RWTH Aachen

\documentclass[twoside,a4paper,article]{combine}


% =========================================================================
\usepackage[latin1]{inputenc}
\usepackage{a4}
\usepackage{fancyhdr}   
%\usepackage{german}    % Uncomment this iff you're writing the report in German
\usepackage{makeidx}
\usepackage{color}
\usepackage{t1enc}		% german letters in the "\hyphenation" - command
\usepackage{latexsym}	% math symbols
\usepackage{amssymb}    % AMS symbol fonts for LaTeX.

\usepackage{graphicx}
\usepackage{pslatex}
\usepackage{ifthen}

\usepackage[T1]{fontenc}
\usepackage{pslatex}

\usepackage{psfrag}
\usepackage{subfigure}
\usepackage{url}

% =========================================================================

\setlength{\oddsidemargin}{3.6pt}
\setlength{\evensidemargin}{22.6pt}
\setlength{\textwidth}{426.8pt}
\setlength{\textheight}{654.4pt}
\setlength{\headsep}{18pt}
\setlength{\headheight}{15pt}
\setlength{\topmargin}{-41.7pt}
\setlength{\topskip}{10pt}
\setlength{\footskip}{42pt}

\setlength{\parindent}{0pt}

% =========================================================================

\graphicspath{
	{pictures/}
}

%%%
% We want also subsubsections to be enumerated
%%%
\setcounter{secnumdepth}{3}
\setcounter{tocdepth}{3}

\makeglossary
%\makeindex

% =========================================================================
\begin{document}

% Template for seminar reports
% Seminar Current Topics in Computer Vision and Machine Learning

\begin{titlepage}


\begin{center}
\ 
\vspace{3.5cm}


\textsf
{
Fakult�t f�r Maschinenwesen\\
Lehrstuhl f�r Computergest�tzte Analyse technischer Systeme\\
German Research School for Simulation Sciences GmbH\\
Simulation Sciences Seminar\\
Univ.-Prof. Marek Behr
}


\rule{\linewidth}{1pt}

\vspace{1.75cm}
\LARGE
\textbf{Master Thesis}

\vspace{1.7cm}
\huge
Simulation of a Quantum Annealer

\vspace{3.0cm}
\Large
Ting-Jui Hsu\\
\large
Matriculation Number: 351218

\vspace{0.5cm}
July 2016

\vspace{1.05cm}
\rule{\linewidth}{1pt}

\vspace{0.5cm}
\textsf{\textbf{
\normalsize
\begin{tabular}{ll}
Advisor: Prof. Dr. Kristel Michielsen\\
\end{tabular}
}}
\end{center}

\end{titlepage}


\begin{abstract}
% +++++++++++++++++++++++++
% Insert your Abstract here
% +++++++++++++++++++++++++
\end{abstract}

\tableofcontents
\newpage
% =========================================================================

% +++++++++++++++++++++++++
% Insert your Text here
% +++++++++++++++++++++++++

% +++++++++++++++++++++++++
% Example: (Detele this)
% +++++++++++++++++++++++++

\section{Introduction}

\section{Quantum Annealer}

\subsection{Time-Dependent Schr�dinger Equation}
\subsection{Quantum Annealing}
\subsection{Adiabatic Theorem}
\subsection{Landau-Zener Theorem}

\section{Optimization Problem}

\subsection{2-SAT Problem}
\subsection{The Mapping of Hamiltonian}
\subsection{Annealing Time scheme}

\section{Quantum Annealer Simulation without Heat Bath}
\subsection{??Simulation Set up}
\subsection{Algorithm}
\subsubsection{Full Diagonalization}
\subsubsection{Suzuki-Trotter Product Formula}
\subsection{Result}
\subsubsection{The Effect of step size tau}
\subsubsection{The Effect of Annealing Time}
\subsubsection{The Effect of Minimum Gap}

\section{Quantum Annealer Simulation with Heat Bath}
\subsection{??Simulation Set up}
\subsection{Algorithm}
\subsubsection{Boltzmann Distribution}
\subsubsection{Random wave function}
\subsection{Result}
\subsubsection{The Effect of Temperature}
\subsubsection{The Process toward Quasi-static}
\subsubsection{The Effect of Annealing Time}
\subsubsection{The Effect of Minimum Gap}

\section{??D-wave Practice}
\section{??Parallelization}
\section{??Applicatoin on Machine Learning / Simulation}


\section{Conclusion}




Result on spin and Energy env,sys,se

success probability of ground state

spin system

energy spectrum

evolution of spin and energy

Runtime comparison

Parallel possibility

-Introduction

-Annealing Theorem

-Full-diagonalization

-suzuki-trotter product formula

-size of tau

-The effect of annealing Time w/ w/o Heat bath

-The effect of minimum gap w/ w/o Heat bath

-Landau-Zener Theorem

-2-SAT problem 

-Random wave function  

-The effect of heat bath(Coherent - Transverse - quasiequilibrium) at different temperature

D-wave practice

-Hamiltonian Mapping

-Annealing Time scheme

-Time-Dependent Schr�dinger Equation

-Boltzmann distribution + trace of the observable
 


%
%\section{Introduction}
%
%Please specify your name, matriculation number, name of advisor and the title of your report in \linebreak
%\verb+titlepage.tex+.
%The title page will not count for the 20 pages.
%
%Using bibtex you can cite in an organized way and without much work.
%Just enter the information about a paper or an article you want to cite in the \verb+seminar_report.bib+ file and use \verb+\cite+ to cite them. For example \cite{Author08CVPR},\cite{Author04IJCV}.
%Don't forget to compile the bib file and Latex will add all the cited references at the end.
%Cite all the literature you use and state where figures are from!
%
%\section{Section}
%\label{section}
%I am a section. Latex will give me a number \emph{automatically} and put me into the table of contents.
%Using \verb+\label+ and \verb+\ref+ you can use Latex to write that this is Section \ref{section}.
%
%\subsection{a subsection}
%I am a subsection.
%
%\begin{itemize}
%\item I am an item.
%\item [-] I am another item.
%\end{itemize}
%
%
%\subsubsection{a small subsection}
%I am a subsubsection, an even smaller subsection.
%
%\begin{tabular}{|l|c|}
%\hline
%I am a tabular & with two columns. \\
%\hline
%The left column is aligned left & and the right columns is centered. \\
%\hline
%\end{tabular}
%
%
%\section{Another Section}
%
%\begin{figure}[h]
%\centering
%\includegraphics[height=100pt]{doge.jpeg}
%\caption{Insert caption here. Image from \cite{lenna}. }
%\label{example_figure}
%\end{figure}
%Figure \ref{example_figure} also gets a number automatically and will be placed where Latex thinks it looks good. You can specify a preference with h(ere), t(op), b(ottom), p(age).
%
%\section{Equations}
%Latex is also really good at printing equations: $E=mc^2$
%
%\begin{equation}
%A = \sum_{i=1}^N A_1 \cdot A_2
%\end{equation}
%

% +++++++++++++++++++++++++

% =========================================================================
\bibliographystyle{alpha}
\bibliography{seminar_report}

% =========================================================================

\end{document}

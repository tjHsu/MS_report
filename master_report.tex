%    Template for seminar reports
% Seminar Current Topics in Computer Vision and Machine Learning
% Summer Semester 2015
% Computer Vision Group, Visual Computing Institute, RWTH Aachen

\documentclass[twoside,a4paper,article]{combine}


% =========================================================================
\usepackage[latin1]{inputenc}
\usepackage{a4}
\usepackage{fancyhdr}   
%\usepackage{german}    % Uncomment this iff you're writing the report in German
\usepackage{makeidx}
\usepackage{color}
\usepackage{t1enc}		% 	german letters in the "\hyphenation" - command
\usepackage{latexsym}	% math symbols
\usepackage{amssymb}    % AMS symbol fonts for LaTeX.
\usepackage{amsmath} 
\usepackage{graphicx}
\usepackage{pslatex}
\usepackage{ifthen}

\usepackage[T1]{fontenc}
\usepackage{pslatex}

\usepackage{psfrag}
\usepackage{subfigure}
\usepackage{url}

% =========================================================================

\setlength{\oddsidemargin}{3.6pt}
\setlength{\evensidemargin}{22.6pt}
\setlength{\textwidth}{426.8pt}
\setlength{\textheight}{654.4pt}
\setlength{\headsep}{18pt}
\setlength{\headheight}{15pt}
\setlength{\topmargin}{-41.7pt}
\setlength{\topskip}{10pt}
\setlength{\footskip}{42pt}

\setlength{\parindent}{0pt}

% =========================================================================

\graphicspath{
	{pictures/}
}

%%%
% We want also subsubsections to be enumerated
%%%
\setcounter{secnumdepth}{3}
\setcounter{tocdepth}{3}

\makeglossary
%\makeindex

% =========================================================================
\begin{document}

% Template for seminar reports
% Seminar Current Topics in Computer Vision and Machine Learning

\begin{titlepage}


\begin{center}
\ 
\vspace{3.5cm}


\textsf
{
Fakult�t f�r Maschinenwesen\\
Lehrstuhl f�r Computergest�tzte Analyse technischer Systeme\\
German Research School for Simulation Sciences GmbH\\
Simulation Sciences Seminar\\
Univ.-Prof. Marek Behr
}


\rule{\linewidth}{1pt}

\vspace{1.75cm}
\LARGE
\textbf{Master Thesis}

\vspace{1.7cm}
\huge
Simulation of a Quantum Annealer

\vspace{3.0cm}
\Large
Ting-Jui Hsu\\
\large
Matriculation Number: 351218

\vspace{0.5cm}
July 2016

\vspace{1.05cm}
\rule{\linewidth}{1pt}

\vspace{0.5cm}
\textsf{\textbf{
\normalsize
\begin{tabular}{ll}
Advisor: Prof. Dr. Kristel Michielsen\\
\end{tabular}
}}
\end{center}

\end{titlepage}


\begin{abstract}
% +++++++++++++++++++++++++
% Insert your Abstract here
% +++++++++++++++++++++++++
\end{abstract}

\tableofcontents
\newpage
% =========================================================================


\section{Introduction}
Quantum annealer is a machine that can perform adiabatic quantum computation. D-wave in fact has already sell their quantum annealing machine to several customers. Although quantum annealing machine can perform quantum computation, it is not universal quantum computer. It specialised in optimisation problem. 

Quantum annealer is built based on spin 1/2 system. Which is Ising-model. The qubit in this system has two state, which is perfect for solving the 2-SAT problems. Start from an initial Hamiltonian, we slowly turn off this initial Hamiltonian and turn on the problem Hamiltonian. If the system is at its ground state and the whole process is slow enough, the system will also end at ground state with the problem Hamiltonian and we get the solution of the 2-SAT problem. 

According to the adiabatic theorem this should work in a pure system, however, in the real case with environment. The system will be coupled with a heat bath. This heat bath may damage the coherent of the system evolution. In my paper, this is the thing i want to simulate.

Start from a 8 spin system. I use two possible algorithm to simulate. Then i move on to a 8 spin system with 8 spin environment. To see the influence of the hear bath, I compare the result in the successful probability. 

In the second section, I will briefly introduce the the quantum annealer and the theorem. In section 3, I will introduce the optimization porblem, which believed to be a potential application of the quantum annealer. In section 4, I first simulate the quantum annealer without the heat bath. In section 5, I then simulate the quantum annealing process with hear bath. In the last section we try to discuss the possibility of using quantum annealing for machine learning.  

\newpage

\section{Quantum Annealer}

\subsection{Time-Dependent Schr�dinger Equation}
\begin{equation*}
i\hbar\frac{\partial \psi}{\partial t}=H\psi
\end{equation*}

\subsection{Quantum Annealing}
QA is a technique to find the ground state of a optimisation problem. different from the thermal annealing is that can tunnel through the barrier. 

\subsection{Adiabatic Theorem}
Once the annealing time is long enough. We should be able to find the ground state.
\subsection{Landau-Zener Theorem}
The probability of finding the ground state is depend on the temperature and the gap.


\section{Optimization Problem}
\subsection{2-SAT Problem}
The 2-SAT problem is made of variables with only two possible option. Among these variables we want to find the best set to get the solution with the lowest cost. 
\subsection{The Mapping of Hamiltonian}

How to map a 2-SAT problem to a Hamiltonian is not a easy question. We using Ising-model to map the 2-SAT problem.
\begin{equation*}
	H=(1-\lambda)H_{init}+\lambda H_{problem}
\end{equation*}
\begin{equation*}
	H_{init}=\sum_{i} h^x_{init} \sigma^x_i
\end{equation*}
\begin{equation*}
	H_{problem}=\sum_{i,j} \{J^x \sigma^x_i \sigma^x_j+J^y \sigma^y_i \sigma^y_j+J^z \sigma^z_i \sigma^z_j \}+ \sum_{i} \{h^x \sigma^x_i+h^y \sigma^y_i+h^z \sigma^z_i\}
\end{equation*}


\subsection{Annealing Time scheme}
	From the previous sector, we can noticed that the time scheme for $\lambda$ should be chosen and set. Here we use linear scheme for $\lambda$.

\section{Quantum Annealer Simulation without Heat Bath}
\subsection{??Simulation Set up}
\subsection{Algorithm}

\subsubsection{Full Diagonalization}
\subsubsection{Suzuki-Trotter Product Formula}
\subsection{Result}
\subsection{the evolution of the spin, energy, and success probability during the annealing}
	\begin{itemize}
		\item \checkmark Figure here: The system energy vs. lambda
		\item \checkmark Figure here: The spin x vs. lambda
		\item \checkmark Figure here: The spin z vs. lambda
		\item \checkmark Figure here: The success probability vs. lambda
	\end{itemize}
\subsubsection{The Effect of step size tau}
	\begin{itemize}
		\item \checkmark Figure here: compare the result of product formula with the result of full diagonalization: success probability vs. annealing Time with different tau
	\end{itemize}
\subsubsection{The Effect of Annealing Time}
	\begin{itemize}
		\item Figure here: compare the result of different annealing time of full diagonalization: success probability vs.lambda with different annealing time
		\item \checkmark Figure here: compare the result of different annealing time of full diagonalization: success probability vs.lambda with different annealing time
		\item \checkmark Figure here: compare the result of different annealing time of full diagonalization: energy vs.lambda with different annealing time
		\item \checkmark Figure here: compare the result of different annealing time of full diagonalization: energy vs.lambda with different annealing time
	\end{itemize}
\subsubsection{The Effect of Minimum Gap}
	\begin{itemize}
		\item \checkmark Figure here: energy spectrum: the energy for all states vs. lambda
		\item \checkmark Figure here: a close view of the result above: the energy for lowest 20~30 states vs. lambda
		\item \checkmark Figure here: compare: the success probability under the annealing time that maximise the minimum gap difference vs. minimum gap value 
	\end{itemize}
	So the scatter effect come from the spin numbering. When the coupling factor become larger and larger. The sccatering effect become more clear. On the other hand, Temperature is not the critical reason to this phenomenon. But the temperature will also enlarge the scatter. The temperature has an influence on how many environment state are part of the evolution. If the temperature is low the revolution only in the ground state. When temperatrue is high, almost all environment state is part of the evolution. the state is average out by the environment states. So the scatter of the problem hamiltonian is not clear. Also since we are in an effective temperature unit. delta gap/Temperature. Therefore, when temperature is larger than gap, they are in excited state. so the annealing is not clear and cant find the ground state. vice versa. when the temperature is smaller than 1. we can see the annealing effect and probability to find the ground state. 
\section{Quantum Annealer Simulation with Heat Bath}
\subsection{??Simulation Set up}
\subsection{Algorithm}
\subsubsection{Boltzmann Distribution}
\subsubsection{Random wave function}
\subsection{Result}
\subsubsection{The Effect of Temperature}
	\begin{itemize}
		\item Figure here: display plots with different coupling factor: success probability of a coupling factor with different temperature vs. annealing Time
	\end{itemize}
\subsubsection{The Process toward Quasi-static}
	\begin{itemize}
		\item \checkmark Figure here: display plots with different temperature: success probability with different coupling factor vs. annealing time
		
		\item \checkmark Figure here: repeat above plots with Random wave function
	\end{itemize}
	
	
\subsubsection{The Effect of Annealing Time}
	This subsection may be removed 
\subsubsection{The Effect of Minimum Gap}
	\begin{itemize}
		\item \checkmark Figure here: compare: the success probability under the annealing time that maximise the minimum gap difference vs. minimum gap value 
	\end{itemize}
\section{??D-wave Practice}
\section{??Parallelization}
\section{??Applicatoin on Machine Learning / Simulation}


\section{Conclusion}
\section{miscellaneous}


\label{section}

Please specify your name, matriculation number, name of advisor and the title of your report in \linebreak
\verb+titlepage.tex+.
The title page will not count for the 20 pages.

Using bibtex you can cite in an organized way and without much work.
Just enter the information about a paper or an article you want to cite in the \verb+seminar_report.bib+ file and use \verb+\cite+ to cite them. For example \cite{Author08CVPR},\cite{Author04IJCV}.
Don't forget to compile the bib file and Latex will add all the cited references at the end.
Cite all the literature you use and state where figures are from!

I am a section. Latex will give me a number \emph{automatically} and put me into the table of contents.
Using \verb+\label+ and \verb+\ref+ you can use Latex to write that this is Section \ref{section}.



\subsection{a subsection}
I am a subsection.

\begin{itemize}
	\item I am an item.
	\item [-] I am another item.
\end{itemize}

\subsubsection{a small subsection}
I am a subsubsection, an even smaller subsection.

\begin{tabular}{|l|c|}
\hline
I am a tabular & with two columns. \\
\hline
The left column is aligned left & and the right columns is centered. \\
\hline
\end{tabular}


\begin{figure}[h]
\centering
\includegraphics[height=100pt]{doge.jpeg}
\caption{Insert caption here. Image from \cite{lenna}. }
\label{example_figure}
\end{figure}
Figure \ref{example_figure} also gets a number automatically and will be placed where Latex thinks it looks good. You can specify a preference with h(ere), t(op), b(ottom), p(age).


%\begin{figure}
%%	\begin{center}
%		\resizebox{!}{!}{\input{pvsgap}}
%%	\end{center}
%\end{figure}

Latex is also really good at printing equations: $E=mc^2$

\begin{equation}
A = \sum_{i=1}^N A_1 \cdot A_2
\end{equation}

\subsubsection{Possible Material}
Result on spin and Energy env,sys,se

success probability of ground state

spin system

energy spectrum

evolution of spin and energy

Runtime comparison

Parallel possibility

-Introduction

-Annealing Theorem

-Full-diagonalization

-suzuki-trotter product formula

-size of tau

-The effect of annealing Time w/ w/o Heat bath

-The effect of minimum gap w/ w/o Heat bath

-Landau-Zener Theorem

-2-SAT problem 

-Random wave function  

-The effect of heat bath(Coherent - Transverse - quasiequilibrium) at different temperature

D-wave practice

-Hamiltonian Mapping

-Annealing Time scheme

-Time-Dependent Schr�dinger Equation

-Boltzmann distribution + trace of the observable


% +++++++++++++++++++++++++

% =========================================================================
\bibliographystyle{alpha}
\bibliography{master_report}

% =========================================================================

\end{document}

%    Template for seminar reports
% Seminar Current Topics in Computer Vision and Machine Learning
% Summer Semester 2015
% Computer Vision Group, Visual Computing Institute, RWTH Aachen

\documentclass[twoside,a4paper,article]{combine}


% =========================================================================
\usepackage[latin1]{inputenc}
\usepackage{a4}
\usepackage{fancyhdr}   
%\usepackage{german}    % Uncomment this iff you're writing the report in German
\usepackage{makeidx}
\usepackage{color}
\usepackage{t1enc}		% 	german letters in the "\hyphenation" - command
\usepackage{latexsym}	% math symbols
\usepackage{amssymb}    % AMS symbol fonts for LaTeX.
\usepackage{amsmath} 
\usepackage{graphicx}
\usepackage{pslatex}
\usepackage{ifthen}

\usepackage[T1]{fontenc}
\usepackage{pslatex}

\usepackage{psfrag}
\usepackage{subfigure}
\usepackage{url}

% =========================================================================

\setlength{\oddsidemargin}{3.6pt}
\setlength{\evensidemargin}{22.6pt}
\setlength{\textwidth}{426.8pt}
\setlength{\textheight}{654.4pt}
\setlength{\headsep}{18pt}
\setlength{\headheight}{15pt}
\setlength{\topmargin}{-41.7pt}
\setlength{\topskip}{10pt}
\setlength{\footskip}{42pt}

\setlength{\parindent}{0pt}

% =========================================================================

\graphicspath{
	{pictures/}
}

%%%
% We want also subsubsections to be enumerated
%%%
\setcounter{secnumdepth}{3}
\setcounter{tocdepth}{3}

\makeglossary
%\makeindex

% =========================================================================
\begin{document}

% Template for seminar reports
% Seminar Current Topics in Computer Vision and Machine Learning

\begin{titlepage}


\begin{center}
\ 
\vspace{3.5cm}


\textsf
{
Fakult�t f�r Maschinenwesen\\
Lehrstuhl f�r Computergest�tzte Analyse technischer Systeme\\
German Research School for Simulation Sciences GmbH\\
Simulation Sciences Seminar\\
Univ.-Prof. Marek Behr
}


\rule{\linewidth}{1pt}

\vspace{1.75cm}
\LARGE
\textbf{Master Thesis}

\vspace{1.7cm}
\huge
Simulation of a Quantum Annealer

\vspace{3.0cm}
\Large
Ting-Jui Hsu\\
\large
Matriculation Number: 351218

\vspace{0.5cm}
July 2016

\vspace{1.05cm}
\rule{\linewidth}{1pt}

\vspace{0.5cm}
\textsf{\textbf{
\normalsize
\begin{tabular}{ll}
Advisor: Prof. Dr. Kristel Michielsen\\
\end{tabular}
}}
\end{center}

\end{titlepage}


\begin{abstract}
% +++++++++++++++++++++++++
% Insert your Abstract here
% +++++++++++++++++++++++++
\end{abstract}

\tableofcontents
\newpage
% =========================================================================


\section{Introduction}
\cite{Ladd2010} General Possible Quantum Computer. 

\cite{Boixo2016} The supremacy.

\cite{Ronnow2014} Detect quantum speed up\\

Adiabatic quantum annealing is a kind of quantum computation achieved by adiabatic evolution and quantum annealer is a device that can perform adiabatic quantum annealing. The idea of quantum computation was brought up in early 80s \cite{Feynman1982}. It indicated that a simulation of quantum phenomena is not always available on an classical computer, because the amount of resource required growth exponentially along with the size of the physical system. Instead, a quantum phenomena should be simulate by a computer making use of quantum property. The amount of resource required by the simulation is only proportional to the size of the physical system. \\

D-wave in fact has already sell their quantum annealing machine to several customers. Although quantum annealing machine can perform quantum computation, it is not universal quantum  computer. It specialised in optimisation problem. \\
\\
Quantum annealer is built based on spin 1/2 system. Which is Ising-model. The qubit in this system has two state, which is perfect for solving the 2-SAT problems. Start from an initial Hamiltonian, we slowly turn off this initial Hamiltonian and turn on the problem Hamiltonian. If the system is at its ground state and the whole process is slow enough, the system will also end at ground state with the problem Hamiltonian and we get th e solution of the 2-SAT problem. \\
\\
According to the adiabatic theorem this should work in a pure system, however, in the real case with environment. The system will be coupled with a heat bath. This heat bath may damage the coherent of the system evolution. In my paper, this is the thing i want to simulate.\\ 
\\
Start from a 8 spin system. I use two possible algorithm to simulate. Then i move on to a 8 spin system with 8 spin environment. To see the influence of the hear bath, I compare the result in the successful probability. \\
\\
In the second section, I will briefly introduce the the quantum annealer and the theorem. In section 3, I will introduce the optimization porblem, which believed to be a potential application of the quantum annealer. In section 4, I first simulate the quantum annealer without the heat bath. In section 5, I then simulate the quantum annealing process with hear bath. In the last section we try to discuss the possibility of using quantum annealing for machine learning.  \\
\\



\newpage

\section{( Theory of ) Quantum Annealer}
\cite{Das2008} 
\cite{Matsuda2009}
\cite{Santoro2006}
\cite{Denchev2015} what is the computational value of finite range tunnelling




\subsection{Quantum Annealing}
\cite{Boixo2014} A compare of quantum annealing and simulated annealing

\cite{QABoixo2016} Multi qubit tunneling

\cite{Jin2013} quantum decoherence


!cite Classical Signature of quantum annealing p2.

The idea of annealing in general is to find the lowest-energy state,  which sometime cast into a problem that try to find the lowest value of a cost function. When performing classic annealing, the system start from a random state in a high temperature. Then the temperature decrease, simulated algorithm choose the new state by some criteria. The system evolve by reducing the energy, and the result is usually end up in a local minima instead of global minima. !!The choose criteria\\
\\
Quantum annealing does not evolve by reducing temperature. The system is affect by two Hamiltonian, one is $H_i$, which is our starting Hamiltonian, and the requirement of this Hamiltonian is that the ground state of this is easy to prepared. On the other hand, $H_p$ is the problem Hamiltonian. We map the problem in to $H_p$. The idea is that, if we start from the ground state of $H_i$, and then the $H_i$ is turned off and the $H_p$ is turned on sufficiently slow. The system will end up in the ground state of $H_p$, which is the solution we want to find.\\
\\
In the ideal case, the classical annealing has some random factor, and the QA does not. Therefore a idea QA will always find the solution, or never, which leads to a bimodal distribution. Classical annealing starts from a random state, on the other hand, may have a uni-modal result. this is consider as a criteria for quantumness. !!Quantum annealing with more than one hundred qubits!! they declare to have this quantum characteristic. but the !!Classical signature of quantum annealing!! claimed that by picking same random state, the CA can also show a bimodal result. \\
\\

!!cite Quantum Computation by Adiabatic Evolution Farhi 2000!!

The adiabatic quantum annealing algorithm is implement as follow:\\
\\
1. An ground state of $H_i$ is construct.
\begin{equation*}
	H_i=\sum_{i=1}^{N}h_i^x \sigma_i^x
\end{equation*}   

2. Construct a Hamiltonian from a given problem.
\begin{equation*}
	H_{problem}=\sum_{i,j} \{J^x \sigma^x_i \sigma^x_j+J^y \sigma^y_i \sigma^y_j+J^z \sigma^z_i \sigma^z_j \}+ \sum_{i} \{h^x \sigma^x_i+h^y \sigma^y_i+h^z \sigma^z_i\}
\end{equation*}   

3. Evolve the system according to the time scheme.
\begin{equation*}
	T:linear 
\end{equation*}

4. Compute Schrodinger eqaution according to the above.\\
\\
5. The final state will be in the ground state of the $H_p$, if annealing T is long enough.\\
\\
6. measure the coefficient value of the state. then we have the solution.\\
\\



QA is a technique to find the ground state of a optimisation problem. different from the thermal annealing is that can tunnel through the barrier. 


\subsubsection{Time-Dependent Schr�dinger Equation}
\begin{equation*}
i\hbar\frac{\partial \psi}{\partial t}=H\psi
\end{equation*}

\subsection{Adiabatic Theorem}
Once the annealing time is long enough. We should be able to find the ground state.
\subsection{Landau-Zener Theorem}
The probability of finding the ground state is depend on the temperature and the gap.

\subsection{Annealing Time scheme}
From the previous sector, we can noticed that the time scheme for $\lambda$ should be chosen and set. Here we use linear scheme for $\lambda$.


\section{Optimisation Problem}
\subsection{In General / 2-SAT Problem}
\cite{Das2008} II A Optimisation and annealing
The 2-SAT problem is made of variables with only two possible option. Among these variables we want to find the best set to get the solution with the lowest cost. 
\subsection{The Mapping of Hamiltonian}

How to map a 2-SAT problem to a Hamiltonian is not a easy question. We using Ising-model to map the 2-SAT problem.
\begin{equation*}
	H=(1-\lambda)H_{init}+\lambda H_{problem}
\end{equation*}
\begin{equation*}
	H_{init}=\sum_{i} h^x_{init} \sigma^x_i
\end{equation*}
\begin{equation*}
	H_{problem}=\sum_{i,j} \{J^x \sigma^x_i \sigma^x_j+J^y \sigma^y_i \sigma^y_j+J^z \sigma^z_i \sigma^z_j \}+ \sum_{i} \{h^x \sigma^x_i+h^y \sigma^y_i+h^z \sigma^z_i\}
\end{equation*}


\section {Simulation Algorithm}
\subsection{Algorithm for Ideal Case}
So here i am going to try out two algorithm. First is full diagonalization, which i use Lapack to solve the eigensystem. Product formula break the evolution into small operation and make the computation of large system possible.
\subsubsection{Full Diagonalization}
Use Lapack to diagonalize the H for $\Psi_{t} = e^{iH\tau} \Psi_{t-\tau}$. 

Check Reference \cite{DeRaedt2004} B.Full Diagonalization Approach

\subsubsection{Suzuki-Trotter Product Formula}
Check Reference \cite{DeRaedt2004}  E. Suzuki-Trotter Product-Formula Algorithms

Break the H to small operation that operate on the state. According to $J^x , J^y, and Jz; h^x, h^y, and h^z$, we need to find the pair or quad pair for the $\Psi$

$\circ \circ \circ ,0, \circ$

$\circ \circ \circ ,1, \circ$

$\circ ,0, \circ ,0, \circ$

$\circ ,0, \circ ,1, \circ$

$\circ ,1, \circ ,0, \circ$

$\circ ,1, \circ ,1, \circ$

\subsection{Algorithm for system with Temperature Effect}
Here since we have in total 16 spins, of course we cannot use product formula algorithm. But we because have temperature term now, which means we may run over several states, therefore the product formula algorithm is not enough for use now. Therefore, we use random sampling method.
\subsubsection{Boltzmann Distribution/ Assemble Average}
???	By adding environment, the temperature will affect how many state should we calculate
We still use product formula. and we use assemble average to calculate each initial state exactly.
\subsubsection{Random wave function}
Reference check! II Theory part.
\cite{Hams2000} 





\section{Quantum Annealer Simulation: Ideal Case}
	There is no ideal system. Environment will always affect the subsystem. However, I start from a simple case, that is a subsystem without environment effect. 
\subsection{??Simulation Set up}
	8 spin without environment spin. With linear time scheme. 

\subsection{Result}
\subsection{the evolution of the spin, energy, and success probability during the annealing}
	A general picture on the annealing behaviour. The expectation value of spin should start from x direction and end up in z direction. During the process y may only have some fluctuation. For the energy of subsystem we expect it to first increase and then decrease to a lower level. We knew the ground state ready, so in the end we expect to see the ground state evolute form 0 to 1, if this is a successful annealing.
	\begin{itemize}
		\item \checkmark Figure here: The system energy vs. lambda
		\item \checkmark Figure here: The spin x vs. lambda
		\item \checkmark Figure here: The spin z vs. lambda
		\item \checkmark Figure here: The success probability vs. lambda
	\end{itemize}
\subsubsection{The Effect of step size tau}
	After having a look at the general properties. Now I try to investigate the size of time step. When the Hamiltonian is time-dependent. The tau should be small enough.  other wise the result would deviate. 
	\begin{itemize}
		\item \checkmark Figure here: compare the result of product formula with the result of full diagonalization: success probability vs. annealing Time with different tau.
	\end{itemize}
\subsubsection{The Effect of Annealing Time}
	Setting up the proper $\tau$. Then I start to simulate the annealing process with different annealing time. The annealing should be long enough to find a ground state. Also I compare the result of full diagonalization and product formula algorithm. 
	\begin{itemize}
		\item Figure here: compare the result of different annealing time of full diagonalization: success probability vs.lambda with different annealing time
		\item \checkmark Figure here: compare the result of different annealing time of full diagonalization: success probability vs.lambda with different annealing time
		\item \checkmark Figure here: compare the result of different annealing time of full diagonalization: energy vs.lambda with different annealing time
		\item \checkmark Figure here: compare the result of different annealing time of full diagonalization: energy vs.lambda with different annealing time
	\end{itemize}
\subsubsection{The Effect of Minimum Gap}
	According to Landau-Zener theorem, the successful probability depend on the gap in between the ground state and the first excited state.
	\begin{itemize}
		\item \checkmark Figure here: energy spectrum: the energy for all states vs. lambda
		\item \checkmark Figure here: a close view of the result above: the energy for lowest 20~30 states vs. lambda
		\item \checkmark Figure here: compare: the success probability under the annealing time that maximise the minimum gap difference vs. minimum gap value 
	\end{itemize}
	
\section{Quantum Annealer Simulation with Temperature Effect}
	After the case without environment, now we move on to the one with environment.
\subsection{??Simulation Set up}
	Here the system consist of a 8-spins subsystem and 8-spins environment. The interaction depend on the coupling factor. When factor is 0, the subsystem cannot be affected by the environment at all. On the other, if the factor is 1, the spin of environment may fully interact with subsystem just as one of the spin inside.

\subsection{Result}
\subsubsection{The Effect of Temperature}
	\begin{itemize}
		\item Figure here: display plots with different coupling factor: success probability of a coupling factor with different temperature vs. annealing Time
	\end{itemize}
\subsubsection{The Process toward Quasi-static}
	See Fig 3 of this reference	\cite{Amin2015} 
	
	In the beginning of the annealing. The probability of finding the ground state may still increase, because the environment haven't yet to affect the subsystem. Then when the environment start interacting with subsystem. After it equilibrium with the subsystem. It will continuous to anneal and the probability will increase. 
	\begin{itemize}
		\item \checkmark Figure here: display plots with different temperature: success probability with different coupling factor vs. annealing time
		
		\item \checkmark Figure here: repeat above plots with Random wave function
	\end{itemize}
	
	
\subsubsection{The Effect of Annealing Time}
	This subsection may be removed 
\subsubsection{The Effect of Minimum Gap}
	So the scatter effect come from the spin numbering. When the coupling factor become larger and larger. The sccatering effect become more clear. On the other hand, Temperature is not the critical reason to this phenomenon. But the temperature will also enlarge the scatter. The temperature has an influence on how many environment state are part of the evolution. If the temperature is low the revolution only in the ground state. When temperatrue is high, almost all environment state is part of the evolution. the state is average out by the environment states. So the scatter of the problem hamiltonian is not clear. Also since we are in an effective temperature unit. delta gap/Temperature. Therefore, when temperature is larger than gap, they are in excited state. so the annealing is not clear and cant find the ground state. vice versa. when the temperature is smaller than 1. we can see the annealing effect and probability to find the ground state. 
	\begin{itemize}
		\item \checkmark Figure here: compare: the success probability under the annealing time that maximise the minimum gap difference vs. minimum gap value.
	\end{itemize}
\section{??D-wave Practice}
	\cite{Johnson2011}
	D-wave use of super conducting qubit
	
\section{??Parallelization}
	\definecolor{light-gray}{gray}{0.4}
	\textcolor{light-gray}{Shoul I put the result of openMP parallerization?}
	It is obvious that i cant put the parallization in between the different time step, because they depend on each other. So i put openmp in side each step for the state. also because the calculation may cause race condition, i have to put it in the second loop instead of the most outside one. this may decrease the optimisation of the running time. However, i think this is the moderate way to get a balance of not getting code too complicate and too slow running time.
	
	! \textcolor{red}{The runtime difference for a single run with core 1,2,4,8,16,32,64,128.}

	
\section{Applicatoin on Machine Learning / Simulation}
	
	\cite{Adachi2015}
	\cite{Benedetti2016}
	\cite{Boyda2017}
	\cite{OMalley2017}
	\cite{Potok2017}
	
	
	Application
	 
	-tree recognition
	
	-hand writing
	
	-face detection
	


\section{Conclusion}
\section{miscellaneous}


\label{section}

Please specify your name, matriculation number, name of advisor and the title of your report in \linebreak
\verb+titlepage.tex+.
The title page will not count for the 20 pages.

Using bibtex you can cite in an organized way and without much work.
Just enter the information about a paper or an article you want to cite in the \verb+seminar_report.bib+ file and use \verb+\cite+ to cite them. For example \cite{Author08CVPR},\cite{Author04IJCV}.
Don't forget to compile the bib file and Latex will add all the cited references at the end.
Cite all the literature you use and state where figures are from!

I am a section. Latex will give me a number \emph{automatically} and put me into the table of contents.
Using \verb+\label+ and \verb+\ref+ you can use Latex to write that this is Section \ref{section}.



\subsection{a subsection}
I am a subsection.

\begin{itemize}
	\item I am an item.
	\item [-] I am another item.
\end{itemize}

\subsubsection{a small subsection}
I am a subsubsection, an even smaller subsection.

\begin{tabular}{|l|c|}
\hline
I am a tabular & with two columns. \\
\hline
The left column is aligned left & and the right columns is centered. \\
\hline
\end{tabular}


\begin{figure}[h]
\centering
\includegraphics[height=100pt]{doge.jpeg}
\caption{Insert caption here. Image from \cite{lenna}. }
\label{example_figure}
\end{figure}
Figure \ref{example_figure} also gets a number automatically and will be placed where Latex thinks it looks good. You can specify a preference with h(ere), t(op), b(ottom), p(age).


%\begin{figure}
%%	\begin{center}
%		\resizebox{!}{!}{\input{pvsgap}}
%%	\end{center}
%\end{figure}

Latex is also really good at printing equations: $E=mc^2$

\begin{equation}
A = \sum_{i=1}^N A_1 \cdot A_2
\end{equation}

\subsubsection{Possible Material}
Result on spin and Energy env,sys,se

success probability of ground state

spin system

energy spectrum

evolution of spin and energy

Runtime comparison

Parallel possibility

-Introduction

-Annealing Theorem

-Full-diagonalization

-suzuki-trotter product formula

-size of tau

-The effect of annealing Time w/ w/o Heat bath

-The effect of minimum gap w/ w/o Heat bath

-Landau-Zener Theorem

-2-SAT problem 

-Random wave function  

-The effect of heat bath(Coherent - Transverse - quasiequilibrium) at different temperature

D-wave practice

-Hamiltonian Mapping

-Annealing Time scheme

-Time-Dependent Schr�dinger Equation

-Boltzmann distribution + trace of the observable


% +++++++++++++++++++++++++

% =========================================================================
\bibliographystyle{alpha}
\bibliography{master_report}

% =========================================================================

\end{document}

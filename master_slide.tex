\documentclass{beamer}
 
\usepackage[utf8]{inputenc}
\graphicspath{
	{pictures/}
}
%\usepackage[latin1]{inputenc}
%\usepackage{a4}
\usepackage{fancyhdr}   
%\usepackage{german}    % Uncomment this iff you're writing the report in German
\usepackage{makeidx}
\usepackage{color}
\usepackage{t1enc}		% 	german letters in the "\hyphenation" - command
\usepackage{latexsym}	% math symbols
\usepackage{amssymb}    % AMS symbol fonts for LaTeX.
\usepackage{amsmath} 
\usepackage{physics}
\usepackage{graphicx}
\usepackage{pslatex}
\usepackage{ifthen}
\usepackage{tcolorbox}

\usepackage[T1]{fontenc}
\usepackage{pslatex}

\usepackage{psfrag}
\usepackage{subfigure}
\usepackage{url}
 
 
%Information to be included in the title page:
\title{Simulation of a Quantum Annealer for Solving the 2-satisfiability problem at Zero and Finite Temperature}
\author{Ting-Jui Hsu}
\institute{Quantum Information Group}
\date{22.06.2017}
 
 
%------------------------------------------------------------------------------------------
\AtBeginSection[]
{
	\begin{frame}
		\frametitle{Table of Contents}
		\tableofcontents[currentsection]
	\end{frame}
}
%------------------------------------------------------------------------------------------ 
 
 
 
\begin{document}
 %------------------------------------------------------------------------------------------
\frame{\titlepage}
%------------------------------------------------------------------------------------------

%------------------------------------------------------------------------------------------
\begin{frame}
	\frametitle{The Main Idea}
	\begin{itemize}
		\item Quantum annealing provides a different approach to solve the 2-SAT optimisation problem.\\
		\item Quantum annealing makes use of quantum fluctuation while simulated annealing is based on thermal fluctuation. \\
		\item The Quantum annealing process can be effected not only by the total annealing time, minimum gap between the ground state and the first excited state of the system, but also the temperature.
	\end{itemize}
\end{frame}

%------------------------------------------------------------------------------------------


%------------------------------------------------------------------------------------------

\begin{frame}
	\frametitle{Table of Contents}
	\tableofcontents
\end{frame}

%------------------------------------------------------------------------------------------


\section{Definitions and Properties}
%%------------------------------------------------------------------------------------------ 
%\begin{frame}
%\frametitle{Sample frame title}
%This is a text in first frame. This is a text in first frame. This is a text in first frame.
%\end{frame}
%%------------------------------------------------------------------------------------------
%%------------------------------------------------------------------------------------------
%\begin{frame}
%	\frametitle{Sample frame title}
%	This is a text in second frame. 
%	For the sake of showing an example.
%	
%	\begin{itemize}
%		\item<1-> Text visible on slide 1
%		\item<2-> Text visible on slide 2
%		\item<3-> Text visible on slide 3
%		\item<4-> Text visible on slide 4
%	\end{itemize}
%	
%\end{frame}
%%------------------------------------------------------------------------------------------
%%------------------------------------------------------------------------------------------
%\begin{frame}
%	\frametitle{Include Doge}
%	\begin{figure}[h]
%		\centering
%		\includegraphics[height=100pt]{doge.jpeg}
%		\caption{Insert caption here.}
%		
%	\end{figure}
%\end{frame}
%%------------------------------------------------------------------------------------------

%------------------------------------------------------------------------------------------
\begin{frame}
	\frametitle{Introdution of the Quantum Annealing}
	\begin{itemize}
		\item 	The goal of quantum annealing process is to find the ground state of the problem Hamiltonian.
		\item   The system is first prepared in the ground state of the initial Hamiltonian. During the annealing, the
		transverse field is slowly turned off while the problem Hamiltonian is slowly turned on. If this procedure
		progress slowly enough, the system will remain in the ground state. At the end of the annealing process,
		the transverse field is complete off and the system should have evolved to the ground state of the problem
		Hamiltonian that encoded the given optimisation problem.
	\end{itemize}

	
	
\end{frame}
%------------------------------------------------------------------------------------------

%------------------------------------------------------------------------------------------
\begin{frame}
	\frametitle{Introdution of the Quantum Annealing}
	Based on Ising model, the Hamiltonian used by quantum annealing process can be written as follow
	
	\begin{equation}
	\label{Hamiltonian_set}
	\begin{split}
	&H(t)=(1-\frac{t}{T} )H_{init}+(\frac{t}{T})H_{problem} \\
	&H_{init}= \sum_{i=1}^{N}h_i^x \sigma_i^x\\
	&H_{problem}= \sum_{i,j}^N J_{ij}^z \sigma^z_i \sigma^z_j + \sum_{i}^N h_i^z \sigma^z_i,\\
	\end{split}
	\end{equation} 
	
	where t is the current timestep and T is toatal annealing time.\\
	
	
	
\end{frame}
%------------------------------------------------------------------------------------------




%------------------------------------------------------------------------------------------
\begin{frame}
	\frametitle{Difference between the Quantum Annealing and Simulated Annealing}
	(I should redraw this figure)
	\begin{figure}[h]
		\centering
		\includegraphics[height=150pt]{qaca.png}
		\caption{The difference between QA and CA. Image from [DC08]. }
		\label{diff_qa_ca}
	\end{figure}
	\footnote{[DC08] Arnab Das and Bikas K. Chakrabarti. Colloquium: Quantum annealing and analog quantum
		computation. Reviews of Modern Physics, 80(3):1061–1081, 2008.}
\end{frame}
%------------------------------------------------------------------------------------------

%------------------------------------------------------------------------------------------
\begin{frame}
	\frametitle{Adiabatic Theorem}
	If a quantum system stays in an eigenstate of a slowly varying Hamiltonian at one time, it will remain in an eigenstate at later times, while its eigenenergy evolves continuously.\\
	\begin{equation*}
	\begin{split}
	H(t)\ket{n(t)}=E_n(t)\ket{n(t)}\\
	-t\frac{\partial}{\partial t}\ket{\Psi (t)} = H(t)\ket{\Psi(t)} \\
	H_D(t)=U^{-1}(t)H(t)U^{1}(t)\\
	H_D\ket{\Psi}_D = i \frac{\partial}{\partial t}\ket{\Psi}_D - i \frac{\partial}{\partial t} U^{-1} \ket{\Psi}\\
	H_D(t)\ket{\Psi(t)}_D = i \frac{\partial}{\partial t} \ket{\Psi(t)}_D
	\end{split}
	\end{equation*}
\end{frame}
%------------------------------------------------------------------------------------------

%------------------------------------------------------------------------------------------
\begin{frame}
	\frametitle{Landau-Zener Transition}
	\begin{itemize}
		\item Landau-Zener formula gives out the probability of a diabatic transition from a lower energy eigenstate to a higher energy eigenstate.
		\item Although adiabatic theorem implies that no diabatic transition will occur if Hamiltonian could vary infinitely slow, this is not
		the case for real system with a finite annealing time.
	\end{itemize}
	The probability of diabatic transition, $P_{diabatic}$, can be described by \\
	\begin{equation*}
	\begin{split}
	P_{diabatic}(T) = \exp({\frac{-2\pi\Delta_{min}^2 T}{\alpha\Gamma_0}}),
	\end{split}
	\end{equation*}
	where $T$ is the total annealing time, $\Delta$ is the minimum gap between two energy eigenstates which are always referred to the gap between the ground state and the first excited state in this report, $\alpha$ is the !relative slope of the two states, and $\Gamma_0$ is the initial strength of the transverse field which is set in $1$ in this report.
\end{frame}
%------------------------------------------------------------------------------------------

%------------------------------------------------------------------------------------------
\begin{frame}
	\frametitle{Combinatorial Optimisation}
	If one need to find the best choice among all available options consist of many independent factors, it can be considered
	as a combinatorial optimisation problem.\\
	A combinatorial optimisation has the following properties:\\
	\begin{itemize}
		\item The optimal solution is searched from a finite set of objects.
		\item This finite set is discrete.
		\item A Brute-force search is usually not feasible.
	\end{itemize}
		
\end{frame}
%------------------------------------------------------------------------------------------

%------------------------------------------------------------------------------------------
\begin{frame}
	\frametitle{The 2-Satisfiability Problem}
	\framesubtitle{General Aspect}
	The boolean satisfiability problem, abbreviated as the SAT problem, is a task of checking whether a given
	set of boolean formula can be satisfied.\\
	A k-SAT problem is a problem with k as the upper limit of the
	number of variables in one clause. \\
	For example, an 2-SAT problem can be written as 
	\begin{equation*}
	(x_1\lor x_2)\land (x_3\lor \neg x_4),
	\end{equation*}
	where $\land$,$\lor$, and $\neg$ state for logical and, logical or, and logical not respectively. 
	
\end{frame}
%------------------------------------------------------------------------------------------

%------------------------------------------------------------------------------------------
\begin{frame}
	\frametitle{The 2-Satisfiability Problem}
	\framesubtitle{Mapping a givent problem to the Hamiltonian}
	A key question is how to map a given problem into the Hamiltonian. \\
	A possible mapping is as follow :
	\begin{table}[h!]
		
		
		
		\begin{center}
			\begin{tabular}{|c|c|c|c|c|}
				
				\multicolumn{5}{c}{2-SAT Variables}\\
				\hline
				& T & T & T & F \\
				\hline
				\hline
				$x_1$ & 1 & 1 & 0 & 0 \\
				\hline
				$x_2$ & 1 & 0 & 1 & 0 \\
				\hline
			\end{tabular}
			\quad
			$\Rightarrow$
			\quad
			\begin{tabular}{|c|c|c|c|c|}
				\multicolumn{5}{c}{Ising variables}\\
				\hline
				& T & T & T & F \\
				\hline
				\hline
				$\sigma_1$ & 1 & 1 & -1 & -1 \\
				\hline
				$\sigma_2$ & 1 & -1 & 1 & -1 \\
				\hline
				$m=\sigma_1+\sigma_2$ & 2 & 0 & 0 & -2 \\
				\hline
			\end{tabular}
		\end{center}
		
		\begin{center}
			\begin{tabular}{|c|c|c|c|c|}
				
				\hline
				& T & T & T & F \\
				\hline
				\hline
				$x_3$ & 1 & 1 & 0 & 0 \\
				\hline
				$x_4$ & 0 & 1 & 0 & 1 \\
				\hline
				
			\end{tabular}
			\quad
			$\Rightarrow$
			\quad
			\begin{tabular}{|c|c|c|c|c|}
				\hline
				& T & T & T & F \\
				\hline
				\hline
				$\sigma_3$ & 1 & 1 & -1 & -1 \\
				\hline
				$\sigma_4$ & -1 & 1 & -1 & 1 \\
				\hline
				$m=\sigma_3-\sigma_2$ & 2 & 0 & 0 & -2 \\
				\hline
			\end{tabular} 
			
		\end{center}
%		\caption{A way to map the 2-SAT clauses to the Ising variables.}
	\end{table} 
	
\end{frame}
%------------------------------------------------------------------------------------------

%------------------------------------------------------------------------------------------
\begin{frame}
	\frametitle{The 2-Satisfiability Problem}
	\framesubtitle{Mapping a givent problem to the Hamiltonian}
	
	In order to encode the solution of the 2-SAT problem into the ground state of the Ising Hamiltonian, an Ising Hamiltonian can be built up with a magnetisation $m$. The Hamiltonian with magnetisation $m$ is given by 
	
	\begin{equation*}
	\begin{split}
	H & = m \cdot (m-2)\\
	& = \sigma_i^2 + \sigma_j^2 + 2\sigma_i \sigma_j -2\sigma_i -2\sigma_j\\
	& = 2\sigma_i \sigma_j -2\sigma_i -2\sigma_j + const.\\
	\end{split}
	\end{equation*}
	
	After comparing with the Hamiltonian mentioned in previous section, one can correspond $h_i^z$, $h_j^z$, and $J_{ij}$ to 2, 2, and -2 respectively. Then one just repeats this procedure for all clauses. 
	%After this mapping is complete, the original 2-SAT problem can be solved with quantum annealing technique, because the ground state of the Ising Hamiltonian can be transformed back to the solution of the 2-SAT problem. \\  
	
	
\end{frame}
%------------------------------------------------------------------------------------------


%------------------------------------------------------------------------------------------
\begin{frame}
	\frametitle{The Quantum Annealing Algorithm}
	\begin{tcolorbox}[title=Quantum Annealing Algorithm]
		Set up $H_{problem}$ according to a given problem. Combine $H_{init}$ and $H_{problem}$ to Build $H(t)$. \\
		%\begin{equation*}
		%	H_{problem}=\sum_{i,j} \{J^x \sigma^x_i \sigma^x_j+J^y \sigma^y_i \sigma^y_j+J^z \sigma^z_i \sigma^z_j \}+ \sum_{i} \{h^x \sigma^x_i+h^y \sigma^y_i+h^z \sigma^z_i\}
		%\end{equation*}   
		
		1. Initialise the system to the ground state of $H_{init}$.\\ %The ground state of $H_{iniy}$ is designed in a way that it is simple to construct.\\
		%\begin{equation*}
		%	H_i=\sum_{i=1}^{N}h_i^x \sigma_i^x
		%\end{equation*}   
		
		
		
		2. Evolve the system by computing time-dependent Schrödinger equation for time $t$ according to a given time scheme. \\
		%\begin{equation*}
		%	T:linear 
		%\end{equation*}
		
		3. The system will end up in the ground state of the $H_p$, if the total annealing T is long enough.\\
		
	\end{tcolorbox} 
\end{frame}
%------------------------------------------------------------------------------------------


\section{Simulation Algorithm}

%------------------------------------------------------------------------------------------
\begin{frame}
	\frametitle{Time-Dependent Schrödinger Equation}
	The whole quantum annealing process can be described by time-dependent Schrödinger equation.\\ 
	Namely, by solving time-dependent Schrödinger equation one can determine the final state of the system.\\
	The solution of this is 
	\begin{equation*}
	\begin{split}
	\Psi(t+\tau)&=U(t+\tau,t)\Psi(t)\\
	&=\exp(i\int_{t}^{t+\tau}H(t)d\tau)\Psi(t),
	\end{split}
	\end{equation*}
	where $\tau$ is the time step, and $U(t+\tau)$ is a unitary matrix that transform system from $t$ to $t+\tau$.\\
	Assume the time step should be small enough to keep $H(t)$ piecewise constant, the solution can be rewritten as 
	\begin{equation*}
	\Psi(t+\tau) =\exp(iH(t)\tau)\Psi(t),
	\end{equation*}
%   In order
%	to conduct numerical computation and ensure the unitary of the evolution operator, time is discretised to
%	small time step τ and H(τ) is assumed to be piecewise constant within these time steps. Therefore, the size
%	of the time step should be small enough to keep H(τ) piecewise. On the other hand, if the size is too small,
%	the computational time will be too long. A proper length should be tested out before going into further
%	simulation.
\end{frame}
%------------------------------------------------------------------------------------------

%------------------------------------------------------------------------------------------
\begin{frame}
	\frametitle{Time-Dependent Schrödinger Equation}
	The $\Psi$ is defined by the direct product state of the N single spin states in z-basis. Each single spin has two possible states, $\ket{0}$ or $\ket{1}$, which correspond to, $\ket{\uparrow}$ or $\ket{\downarrow}$ . The relation between them is defined as 	
	\begin{equation*}
	\ket{0}=\ket{\uparrow}=\left(\begin{array}{c} 1\\0 \end{array}\right), \ket{1}=\ket{\downarrow}=\left(\begin{array}{c} 0\\1 \end{array}\right)
	\end{equation*}
	
	Thus, the single spin state can be written as a linear superposition of theses two basis state, 	
	\begin{equation*}
	\ket{\psi} = a(0)\ket{0} + a(1)\ket{1},
	\end{equation*}
	
	where $a(0)$ and $a(1)$ are the coefficient of the amplitude of each state. Furthermore, a N-spin system is made up of $2^N$ state vectors and it can be written as 	
	\begin{equation*}
	\label{psi_in_compuationalbasis}
	\Psi = a(00\dots0)\ket{00\dots0}+a(00\dots1)\ket{00\dots1}+\dots+a(11\dots0)\ket{11\dots0}+a(11\dots1)\ket{11\dots1},
	\end{equation*}
	where $a(00\dots0), a(00\dots1)\dots, a(11\dots1)$ are the coefficients of each components. 
\end{frame}
%------------------------------------------------------------------------------------------



%------------------------------------------------------------------------------------------
\begin{frame}
	\frametitle{Hamiltonian for a System at Zero Temperature}
	(I should remove this page.)
\end{frame}
%------------------------------------------------------------------------------------------

%------------------------------------------------------------------------------------------
\begin{frame}
	\frametitle{The Full Diagonalisation method}
	\begin{itemize}
		\item Hamiltonian H is an Hermitian operator formed by a $2^N \times 2^N$ matrix.
		\item H can be transformed into a diagonal matrix $\Lambda$ spanned by its eigenvalues and the unitary matrix $V$ of its eigenvectors.
		\item Thus the time evolution can be calculated as $U(\tau) = exp(-i\tau H) = V exp(-i\tau \Lambda) V^\dagger$.
		\item This approach serves mostly as a tool to valid the correctness of other algorithm when solving with time-dependent Schrödinger
	\end{itemize}
%	Hamiltonian H is an Hermitian operator formed by a $2^N \times 2^N$ matrix it has a complete set of eigenvectors and eigenvalues. This implies that the matrix H can be transformed into a diagonal matrix $\Lambda$ spanned by its eigenvalues and the unitary matrix $V$ of its eigenvectors. The transformation is given by $H = V\Lambda V^\dagger$. Thus the time evolution can be calculated as $U(\tau) = exp(-i\tau H) = V exp(-i\tau \Lambda) V^\dagger$. If $V$ and $\Lambda$ are obtained by diagonalising H, time evolution becomes a series of simple matrix multiplication. Since diagonalisation is a typical matrix problem, there are already standard software library can be made use of. \\
%	
\end{frame}
%------------------------------------------------------------------------------------------

%------------------------------------------------------------------------------------------
\begin{frame}
	\frametitle{The Suzuki-Trotter Product Formula Approach}
	\begin{itemize}
		\item The Suzuki-Trotter product formula approach can handle a larger quantum system than the full diagonalisation method does.
		\item Suzuki-Trotter product formula approximate a unitary matrix exponentials by decomposing the matrix properly, that is 
		\begin{equation*}
		\label{suzuki-trotter product formula}
		\begin{split}
		U(t) &= \exp(-i\tau H)\\
		&=\exp(-t\tau(H_1+H_2+\dots+H_K))\\
		&=\lim_{m \to \infty}(\prod_{k=1}^{K}\exp(-i\tau H_k/m))^m
		\end{split}
		\end{equation*}
		\item  The crucial step of this approach is to choose $H_i$ properly in a way that it is easy enough to calculate there matrix exponential efficiently.
	\end{itemize}
\end{frame}
%------------------------------------------------------------------------------------------


%------------------------------------------------------------------------------------------
\begin{frame}
	\frametitle{The Suzuki-Trotter Product Formula Approach}
	Based on the Hamiltonian used in this study, one way to construct the (second order) approximation is 
	
	\begin{equation*}
	\label{tilde_U}
	\begin{split}
	\tilde{U}(\tau) = \exp(\frac{-i\tau H_{\sigma_\alpha}}{2})\exp(\frac{-i\tau H_{\sigma_x \sigma_x}}{2})\exp(\frac{-i\tau H_{\sigma_y \sigma_y}}{2})\exp(-i\tau H_{\sigma_z \sigma_z})\\\exp(\frac{-i\tau H_{\sigma_y \sigma_y}}{2})\exp(\frac{-i\tau H_{\sigma_x \sigma_x}}{2})\exp(\frac{-i\tau H_{\sigma_\alpha}}{2})
	\end{split}
	\end{equation*}

\end{frame}
%------------------------------------------------------------------------------------------







%------------------------------------------------------------------------------------------
\begin{frame}
	\frametitle{Hamiltonian for a System at Finite Temperature}
	In order to simulate a system at finite temperature, the quantum subsystem S is coupled to a quantum environment E and the Hamiltonian of the entire system (i.e. S+E) is defined as
	
	\begin{equation*}
	H = H_S + H_E + GH_{SE},
	\end{equation*} 
	
	where $H_S$ is the Hamiltonian of the subsystem, $H_E$ is the Hamiltonian of the environment, and $H_{SE}$ is the interaction between the subsystem S and the environment E with $G$ indicates the global coupling strength between S and E. 
\end{frame}
%------------------------------------------------------------------------------------------

%------------------------------------------------------------------------------------------
\begin{frame}
	\frametitle{Canonical-Thermal State}
\end{frame}
%------------------------------------------------------------------------------------------

%------------------------------------------------------------------------------------------
\begin{frame}
	\frametitle{The Random Sampling Method}
\end{frame}
%------------------------------------------------------------------------------------------



\section{Simulation Result}
%------------------------------------------------------------------------------------------
\begin{frame}
		\frametitle{Validate Suzuki-Trotter Product Formula Approach with Full Diaganolisation Method}
		\begin{figure}[h]
			\centering
			\includegraphics[height=150pt]{result_picture/compare_full_product/result_1e0.eps}
			\caption{Insert caption here.}
			
		\end{figure}
\end{frame}

%------------------------------------------------------------------------------------------

%------------------------------------------------------------------------------------------
\begin{frame}
	\frametitle{Validate Suzuki-Trotter Product Formula Approach with Full Diaganolisation Method}
	\begin{figure}[h]
		\centering
		\includegraphics[height=150pt]{result_picture/compare_full_product/result_1e-1.eps}
		\caption{Insert caption here.}
		
	\end{figure}
\end{frame}

%------------------------------------------------------------------------------------------

%------------------------------------------------------------------------------------------
\begin{frame}
	\frametitle{Validate Suzuki-Trotter Product Formula Approach with Full Diaganolisation Method}
	\begin{figure}[h]
		\centering
		\includegraphics[height=150pt]{result_picture/compare_full_product/result_1e-2.eps}
		\caption{Insert caption here}
		
	\end{figure}
\end{frame}

%------------------------------------------------------------------------------------------

%------------------------------------------------------------------------------------------
\begin{frame}
	\frametitle{Simulation Result: Ground Energy}
	\begin{figure}
		\centering
		\includegraphics[height=200pt]{result_picture/energy_evolution_product_formula/with_product_formula/Figure_productf_EnergyvsLambda.png}
		\caption{Insert Caption here}
	\end{figure}
\end{frame}
%------------------------------------------------------------------------------------------

%------------------------------------------------------------------------------------------
\begin{frame}
	\frametitle{Simulation Result: Success Probability}
	(should keep only one data here!)
	\begin{figure}
		\centering
		\includegraphics[height=200pt]{result_picture/probability_spin_system/Figure_Annealing_probability.png}

		\caption{Insert Caption here}
	\end{figure}
\end{frame}
%------------------------------------------------------------------------------------------

%------------------------------------------------------------------------------------------
\begin{frame}
	\frametitle{Simulation Result: Spin Value}
	\begin{figure}
		\centering
		\includegraphics[height=180pt]{result_picture/probability_spin_system/Figure_Annealing_spin.png}
		\caption{Insert Caption here}
	\end{figure}
\end{frame}
%------------------------------------------------------------------------------------------

%------------------------------------------------------------------------------------------
\begin{frame}
	\frametitle{Effect of Total Annealing Time}
	\begin{figure}
		\centering
		\includegraphics[height=200pt]{result_picture/probability_spin_system/Figure_Annealing_probability.png}
		
		\caption{Insert Caption here}
	\end{figure}
\end{frame}
%------------------------------------------------------------------------------------------

%------------------------------------------------------------------------------------------
\begin{frame}
	\frametitle{Effect of Minimum Gap}
	\begin{figure}
		\centering
		\includegraphics[height=200pt]{result_picture/energy_evolution_product_formula/with_product_formula/Figure_energy_spectrum.png}
		
		\caption{Insert Caption here}
	\end{figure}
\end{frame}
%------------------------------------------------------------------------------------------

%------------------------------------------------------------------------------------------
\begin{frame}
	\frametitle{Effect of Minimum Gap}
	\begin{figure}
		\centering
		\includegraphics[height=200pt]{result_picture/energy_evolution_product_formula/with_product_formula/Figure_closelook_energy_spectrum.png}
		
		\caption{Insert Caption here}
	\end{figure}
\end{frame}
%------------------------------------------------------------------------------------------

%------------------------------------------------------------------------------------------
\begin{frame}
	\frametitle{Effect of Minimum Gap}
	! I should pickup the first picture to demonstrate the effect of minimum gap for different problem set
	\begin{figure}
		\centering
		\includegraphics[height=150pt]{result_picture/gap_result_environment/Figure_PvsGap_T2e2_Temp1.png}
		
		\caption{Insert Caption here}
	\end{figure}
\end{frame}
%------------------------------------------------------------------------------------------

%------------------------------------------------------------------------------------------
\begin{frame}
	\frametitle{Validate Random Sampling Method}
	\begin{figure}
		\centering
		\includegraphics[width=0.475\textwidth]{result_picture/JvsT_plot/T1/JvsTr1.png}
		\hfill
		\includegraphics[width=0.475\textwidth]{result_picture/JvsT_plot/T1/JvsTwTemp1.png}		
		
		\caption{Insert Caption here}
	\end{figure}
\end{frame}
%------------------------------------------------------------------------------------------

%------------------------------------------------------------------------------------------
\begin{frame}
	\frametitle{Validate Random Sampling Method}
	\begin{figure}
		\centering
		\includegraphics[width=0.475\textwidth]{result_picture/JvsT_plot/T1/JvsTr2.png}
		\hfill
		\includegraphics[width=0.475\textwidth]{result_picture/JvsT_plot/T1/JvsTwTemp1.png}		
		
		\caption{Insert Caption here}
	\end{figure}
\end{frame}
%------------------------------------------------------------------------------------------


%------------------------------------------------------------------------------------------
\begin{frame}
	\frametitle{Validate Random Sampling Method}
	\begin{figure}
		\centering
		\includegraphics[width=0.475\textwidth]{result_picture/JvsT_plot/T1/JvsTr3.png}
		\hfill
		\includegraphics[width=0.475\textwidth]{result_picture/JvsT_plot/T1/JvsTwTemp1.png}		
		
		\caption{Insert Caption here}
	\end{figure}
\end{frame}
%------------------------------------------------------------------------------------------

%------------------------------------------------------------------------------------------
\begin{frame}
	\frametitle{Validate Random Sampling Method}
	\begin{figure}
		\centering
		\includegraphics[width=0.475\textwidth]{result_picture/JvsT_plot/T1/JvsT10runs.png}
		\hfill
		\includegraphics[width=0.475\textwidth]{result_picture/JvsT_plot/T1/JvsTwTemp1.png}		
		
		\caption{Insert Caption here}
	\end{figure}
\end{frame}
%------------------------------------------------------------------------------------------

%------------------------------------------------------------------------------------------
\begin{frame}
	\frametitle{Effect of Total Annealing Time at different Temperature}
	A Quick look\\
	Temperature = 0.02, 1, 1000
	\begin{figure}
		\centering
		\includegraphics[width=0.320\textwidth]{result_picture/JvsT_plot/Exact/JvsTwTemp2e-2.png}
		\hfill
		\includegraphics[width=0.320\textwidth]{result_picture/JvsT_plot/Exact/JvsTwTemp1.png}
		\hfill
		\includegraphics[width=0.320\textwidth]{result_picture/JvsT_plot/Exact/JvsTwTemp1e3.png}
		\caption{Insert Caption here}
	\end{figure}
\end{frame}
%------------------------------------------------------------------------------------------

%------------------------------------------------------------------------------------------
\begin{frame}
	\frametitle{Effect of Total Annealing Time at different Temperature}
	Temperature = 1\\
	Coherent - Non-equilibrium - Quasistatic
	\begin{figure}
		\centering
		\includegraphics[height=150pt]{result_picture/JvsT_plot/Exact/JvsTwTemp1.png}
		\caption{Insert Caption here}
	\end{figure}
\end{frame}
%------------------------------------------------------------------------------------------

%------------------------------------------------------------------------------------------
\begin{frame}
	\frametitle{Effect of Total Annealing Time at different Temperature}
	Temperature = 1000\\
	Coherent - Non-equilibrium -X- Quasistatic
	\begin{figure}
		\centering
		\includegraphics[height=150pt]{result_picture/JvsT_plot/Exact/JvsTwTemp1e3.png}
		\caption{Insert Caption here}
	\end{figure}
\end{frame}
%------------------------------------------------------------------------------------------

%------------------------------------------------------------------------------------------
\begin{frame}
	\frametitle{Effect of Minimum Gap under different Coupling Strength }

	\begin{figure}
		\centering
		\includegraphics[height=150pt]{result_picture/gap_result_environment/Figure_PvsGap_T2e2_Temp1.png}
		
		\caption{Insert Caption here}
	\end{figure}
\end{frame}
%------------------------------------------------------------------------------------------

%------------------------------------------------------------------------------------------
\begin{frame}
	\frametitle{Effect of Minimum Gap under different Temperature }
	
	\begin{figure}
		\centering
		\includegraphics[height=150pt]{result_picture/gap_result_avoid_duplicate/gap_result_temperatur_G02/avoided_duplicate/Figure2_PvsGap_T2e2_G02.png}	
		\caption{Insert Caption here}
	\end{figure}
\end{frame}
%------------------------------------------------------------------------------------------





%%%%%%%%%%%%%%
%%%%%%%%%%%%%%
%%%%%%%%%%%%%%
\end{document}


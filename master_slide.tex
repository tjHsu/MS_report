\documentclass{beamer}
%\documentclass[notes=only]{beamer} 
%\documentclass[notes]{beamer}     
\setbeamertemplate{footline}[frame number]
\beamertemplatenavigationsymbolsempty

\graphicspath{
	{pictures/}
}

%\usepackage[latin1]{inputenc}
\usepackage[utf8]{inputenc}
%\usepackage{a4}
\usepackage{fancyhdr}   
%\usepackage{german}    % Uncomment this iff you're writing the report in German
\usepackage{makeidx}
\usepackage{color}
\usepackage{t1enc}		% 	german letters in the "\hyphenation" - command
\usepackage{latexsym}	% math symbols
\usepackage{amssymb}    % AMS symbol fonts for LaTeX.
\usepackage{amsmath} 
\usepackage{physics}
\usepackage{graphicx}
\usepackage{pslatex}
\usepackage{ifthen}
\usepackage{tcolorbox}


\usepackage[T1]{fontenc}
\usepackage{pslatex}

\usepackage{psfrag}
\usepackage{subfigure}
\usepackage{url}

\usepackage{appendixnumberbeamer}

\usepackage{biblatex}
\addbibresource{master_report.bib}


 
%Information to be included in the title page:
\title{Simulation of a Quantum Annealer for Solving the 2-Satisfiability Problem at Zero and Finite Temperature}
\author{Ting-Jui Hsu}
\institute{Quantum Information Group}
\date{27.06.2017}
  
%------------------------------------------------------------------------------------------
\AtBeginSection[]
{
	\begin{frame}
		\frametitle{Table of Contents}
		\tableofcontents[currentsection]
	\end{frame}
}
%------------------------------------------------------------------------------------------ 

\begin{document}
 %------------------------------------------------------------------------------------------
\frame{\titlepage}
%------------------------------------------------------------------------------------------

%%------------------------------------------------------------------------------------------
%\begin{frame}
%	\frametitle{The Main Idea}
%	\begin{itemize}
%		\item Quantum annealing provides a different approach to solve the 2-SAT optimisation problem.\\
%		\item Quantum annealing makes use of quantum fluctuation while simulated annealing is based on thermal fluctuation. \\
%		\item The Quantum annealing process can be effected not only by the total annealing time, minimum gap between the ground state and the first excited state of the system, but also the temperature.
%	\end{itemize}
%\end{frame}
%
%%------------------------------------------------------------------------------------------

%------------------------------------------------------------------------------------------

\begin{frame}
	\frametitle{Table of Contents}
	\tableofcontents
\end{frame}

%------------------------------------------------------------------------------------------
\note{Hello, so today my topic is about the Simulation of a Quantum Annealer for 2-satisfiability problem. A quick explaination here is that we use adiabatic quantum annealing process to solve a 2-satifiability problem on a quantum annealer. And the for the case at zero temperature, which is the ideal case. and at Finite temperature, which means the system is coupled with the heat bath and the effect of the temperature will be provided by the heat bath to the system. \\
	A general idea that I will mention in the talk is first 
	\begin{itemize}
		\item Quantum annealing provides a different approach to solve the 2-SAT optimisation problem.\\
		\item Quantum annealing makes use of quantum fluctuation while simulated annealing is based on thermal fluctuation. \\
		\item The Quantum annealing process can be effected not only by the total annealing time, minimum gap between the ground state and the first excited state of the system, but also the temperature.
	\end{itemize}
	}


%\section{Definitions and Properties}
\section{Inroduction of the Quantum Annealing}
%%------------------------------------------------------------------------------------------ 
%\begin{frame}
%\frametitle{Sample frame title}
%This is a text in first frame. This is a text in first frame. This is a text in first frame.
%\end{frame}
%%------------------------------------------------------------------------------------------
%%------------------------------------------------------------------------------------------
%\begin{frame}
%	\frametitle{Sample frame title}
%	This is a text in second frame. 
%	For the sake of showing an example.
%	
%	\begin{itemize}
%		\item<1-> Text visible on slide 1
%		\item<2-> Text visible on slide 2
%		\item<3-> Text visible on slide 3
%		\item<4-> Text visible on slide 4
%	\end{itemize}
%	
%\end{frame}
%%------------------------------------------------------------------------------------------
%%------------------------------------------------------------------------------------------
%\begin{frame}
%	\frametitle{Include Doge}
%	\begin{figure}[h]
%		\centering
%		\includegraphics[height=100pt]{doge.jpeg}
%		\caption{Insert caption here.}
%		
%	\end{figure}
%\end{frame}
%%------------------------------------------------------------------------------------------

%------------------------------------------------------------------------------------------
\begin{frame}
	\frametitle{Introdution of the Quantum Annealing}
	\begin{itemize}
		\item The goal of the quantum annealing is to find the ground state of a given Hamiltonian.
		\item The Hamiltonian used by a quantum annealing process can be written as follow
		\begin{equation*}
		\label{Hamiltonian_set}
		\begin{split}
		&H(t)=(1-\frac{t}{T} )H_{init}+(\frac{t}{T})H_{problem} \\
		&H_{init}= -\sum_{i=1}^{N}h_i^x \sigma_i^x\\
		&H_{problem}= -\sum_{i,j}^N J_{ij}^z \sigma^z_i \sigma^z_j -\sum_{i}^N h_i^z \sigma^z_i,\\
		\end{split}
		\end{equation*} 
		where $t$ is the current time step and $T$ is the total annealing time.\\
	\end{itemize}
\end{frame}
%------------------------------------------------------------------------------------------

\note[itemize]{
	\item What a quantum annealer can do is adiabatic quantum annealing.
	\item Ising model
	\item So how can a quantum annealer achieve this? The process can be decribed as follow.
	\item The system is first prepared in the ground state of the initial Hamiltonian. 
	\item During the annealing, the	transverse field is slowly turned off while the problem Hamiltonian is slowly turned on. 
	\item If this procedure	progress slowly enough, the system will remain in the ground state. 
	\item At the end of the annealing process, the transverse field is complete off and the system should have evolved to the ground state of the problem Hamiltonian that encoded the given optimisation problem.
	}

%%------------------------------------------------------------------------------------------
%\begin{frame}
%	\frametitle{Introdution of the Quantum Annealing}
%	\begin{itemize}
%		\item 	The goal of quantum annealing process is to find the ground state of the problem Hamiltonian.
%		\item   The system is first prepared in the ground state of the initial Hamiltonian. During the annealing, the
%		transverse field is slowly turned off while the problem Hamiltonian is slowly turned on. If this procedure
%		progress slowly enough, the system will remain in the ground state. At the end of the annealing process,
%		the transverse field is complete off and the system should have evolved to the ground state of the problem
%		Hamiltonian that encoded the given optimisation problem.
%	\end{itemize}
%\end{frame}
%%------------------------------------------------------------------------------------------

%------------------------------------------------------------------------------------------
\begin{frame}
	\frametitle{The Difference between the Quantum Annealing and the Simulated Annealing}
%	(I should redraw this figure)
	\begin{figure}[h]
		\centering
		\includegraphics[height=150pt]{qaca.png}
		\caption{An illustration shows the difference between the idea of the quantum annealing and the thermal annealing. Image from [DC08]\footnotemark[1]. }
		\label{diff_qa_ca}
	\end{figure}
	\footnotetext[1]{[DC08] Arnab Das and Bikas K. Chakrabarti. Colloquium: Quantum annealing and analog quantum computation. Reviews of Modern Physics, 2008.}
\end{frame}
%------------------------------------------------------------------------------------------

%------------------------------------------------------------------------------------------
\begin{frame}
	\frametitle{Adiabatic Theorem}
%	\begin{itemize}
			If a quantum system stays in an eigenstate of a slowly varying Hamiltonian at one time, it will remain in an eigenstate at later times, while its eigenenergy evolves continuously.\\
%		\item 	A quantum system that evolves unitaryily through a time-dependent Schrödinger equation can be describe by 
%		\begin{equation*}
%		\begin{split}
%		-i\ket{\dot{\Psi}(t)} = H(t)\ket{\Psi(t)},
%		\end{split}
%		\end{equation*}
%		with $\hbar = 1$. Thus, a basis of eigenenergies can be defined by 
%		\begin{equation*}
%		\begin{split}
%		H(t)\ket{n(t)}=E_n(t)\ket{n(t)}.
%		\end{split}
%		\end{equation*}		
%		Since $H(t)$ is diagnolisable, we can have the following transformation.
%		\begin{equation*}
%		\begin{split}
%		H_D(t)=U^{-1}(t)H(t)U^{1}(t)\\
%		H_D\ket{\Psi}_D = i \ket{\dot{\Psi}}_D - i \dot{U^{-1}} \ket{\Psi}\\
%		H_D(t)\ket{\Psi(t)}_D = i \ket{\dot{\Psi}(t)}_D,
%		\end{split}
%		\end{equation*}
%		where $\ket{\Psi}_D \equiv U^{-1}\ket{\Psi}$.
%	\end{itemize}


\end{frame}
%------------------------------------------------------------------------------------------

\note{A physical system remains in its instantaneous eigenstate if a given perturbation is acting on it slowly enough and if there is a gap between the eigenvalue and the rest of the Hamiltonian's spectrum\footnotemark[1]. \\

This is the basis for quantum annealing, because it ensures that once the annealing time is infinitely long, the system will stay in its eigenstate during the annealing process. 
	
Based on Distributive property. 
\footnotetext{ M. Born and V. A. Fock (1928). "Beweis des Adiabatensatzes". Zeitschrift für Physik A. 51 (3–4): 165–180. Bibcode:1928ZPhy...51..165B. doi:10.1007/BF01343193.}
}
%	\begin{equation*}
%	\begin{split}
%	-i\frac{\partial}{\partial t}\ket{\Psi (t)} = H(t)\ket{\Psi(t)} \\
%	H(t)\ket{n(t)}=E_n(t)\ket{n(t)}\\
%	H_D(t)=U^{-1}(t)H(t)U^{1}(t)\\
%	H_D\ket{\Psi}_D = i \frac{\partial}{\partial t}\ket{\Psi}_D - i \frac{\partial}{\partial t} U^{-1} \ket{\Psi}\\
%	H_D(t)\ket{\Psi(t)}_D = i \frac{\partial}{\partial t} \ket{\Psi(t)}_D
%	\end{split}
%	\end{equation*}

%------------------------------------------------------------------------------------------
\begin{frame}
	\frametitle{Landau-Zener Transition}
	\begin{itemize}
		\item The Landau-Zener formula gives out the probability of a diabatic transition from a lower energy eigenstate to a higher energy eigenstate.

	\end{itemize}
	The probability of a adiabatic transition, $P_{adiabatic}$, can be described by 
	\begin{equation*}
	\begin{split}
	&P_{adiabatic}(T) = 1-P_{diabatic}(T)\\
	&P_{diabatic}(T) = \exp(-c\cdot \Delta_{min}^2).
	\end{split}
	\end{equation*}
	\begin{figure}
		\centering
		\includegraphics[height=140pt]{result_picture/energy_evolution_product_formula/with_product_formula/Figure_Landau_Zener.eps}
		
		% \caption{Insert Caption here}
	\end{figure}
	
%	Therefore, the probability of a adiabatic transition states as 
%	\begin{equation*}
%	P_{adiabatic}(T) = 1-P_{diabatic}(T).
%	\end{equation*}
\end{frame}
%------------------------------------------------------------------------------------------

\note{ Although adiabatic theorem implies that no diabatic transition will occur if Hamiltonian could vary infinitely slow, this is not	the case for real system with a finite annealing time. For the annealing with finite annealing time, Landau-Zener formula provides a good model to describe how the system transit between two different state.\\
	Where $T$ is the total annealing time, $\Delta$ is the minimum gap between two energy eigenstates which are always referred to the gap between the ground state and the first excited state in this report, $\alpha$ is the !relative slope of the two states, and $\Gamma_0$ is the initial strength of the transverse field which is set in $1$ in this report.
	}
%	The probability of diabatic transition, $P_{diabatic}$, can be described by \\
%	\begin{equation*}
%	\begin{split}
%	P_{diabatic}(T) = \exp({\frac{-2\pi\Delta_{min}^2 T}{\alpha\Gamma_0}}),
%	\end{split}
%	\end{equation*}
%	where $T$ is the total annealing time, $\Delta$ is the minimum gap between two energy eigenstates which are always referred to the gap between the ground state and the first excited state in this report, $\alpha$ is the !relative slope of the two states, and $\Gamma_0$ is the initial strength of the transverse field which is set in $1$ in this report.

\section{The 2-Satisfiability Problem}
%%------------------------------------------------------------------------------------------
%\begin{frame}
%	\frametitle{Combinatorial Optimisation}
%	If one need to find the best choice among all available options consist of many independent factors, it can be considered
%	as a combinatorial optimisation problem.\\
%	A combinatorial optimisation has the following properties:\\
%	\begin{itemize}
%		\item The optimal solution is searched from a finite set of objects.
%		\item This finite set is discrete.
%		\item A Brute-force search is usually not feasible.
%	\end{itemize}
%		
%\end{frame}
%%------------------------------------------------------------------------------------------

%------------------------------------------------------------------------------------------
\begin{frame}
	\frametitle{The 2-Satisfiability Problem}

	A $k$-SAT problem is a problem with $k$ as the upper limit of the
	number of variables in one clause. \\
	An example 2-SAT problem with 8 boolean variables is shown below: %can be written as follows: 
	\begin{tcolorbox}[title=Example]
	\begin{equation*}
	(x_1\lor x_2)\land (x_3\lor \neg x_4)\land(\neg x_5 \lor x_4)\land(\neg x_6 \lor \neg x_7)\land(\neg x_3 \lor \neg x_8),%\land(\neg x_5 \lor x_7)\land(x_5\lor \neg x_2)
	%sol= 01011110
	\end{equation*}
	where $\land$,$\lor$, and $\neg$ state for logical and, logical or, and logical not respectively. 
	\end{tcolorbox}
\end{frame}
%------------------------------------------------------------------------------------------

\note{	
		2-satisfiability problem is a kind of combinatorial opimisation problem. What is a combinatorial optimisation? 
		If one need to find the best choice among all available options consist of many independent factors, it can be considered
		as a combinatorial optimisation problem.\\
		A combinatorial optimisation has the following properties:\\
		\begin{itemize}
			\item The optimal solution is searched from a finite set of objects.
			\item This finite set is discrete.
			\item A Brute-force search is usually not feasible.
		\end{itemize}
	The boolean satisfiability problem, abbreviated as the SAT problem, is a task of checking whether a given
	set of boolean formula can be satisfied.\\
}

%------------------------------------------------------------------------------------------
\begin{frame}
	\frametitle{The 2-Satisfiability Problem}
%	\framesubtitle{Mapping a givent problem to the Hamiltonian}
	\begin{equation*}
	(x_1\lor x_2)\land (x_3\lor \neg x_4)\land(\neg x_5 \lor x_4)\land(\neg x_6 \lor \neg x_7)\land(\neg x_3 \lor \neg x_8)%\land(\neg x_5 \lor x_7)\land(x_5\lor \neg x_2)
	%sol= 01011110
	\end{equation*}
	A key question is how to map a given problem into the Hamiltonian. \\
	For example, a possible mapping is as follows :
	\begin{table}[h!]
		
		
		
		\begin{center}
			\begin{tabular}{|c|c|c|c|c|}
				
				\multicolumn{5}{c}{2-SAT Variables}\\
				\hline
				& T & T & T & F \\
				\hline
				\hline
				$x_1$ & 1 & 1 & 0 & 0 \\
				\hline
				$x_2$ & 1 & 0 & 1 & 0 \\
				\hline
			\end{tabular}
			\quad
			$\Rightarrow$
			\quad
			\begin{tabular}{|c|c|c|c|c|}
				\multicolumn{5}{c}{Ising variables}\\
				\hline
				& T & T & T & F \\
				\hline
				\hline
				$\sigma_1$ & 1 & 1 & -1 & -1 \\
				\hline
				$\sigma_2$ & 1 & -1 & 1 & -1 \\
				\hline
				$m=\sigma_1+\sigma_2$ & 2 & 0 & 0 & -2 \\
				\hline
			\end{tabular}
		\end{center}
		
		\begin{center}
			\begin{tabular}{|c|c|c|c|c|}
				
				\hline
				& T & T & T & F \\
				\hline
				\hline
				$x_3$ & 1 & 1 & 0 & 0 \\
				\hline
				$x_4$ & 0 & 1 & 0 & 1 \\
				\hline
				
			\end{tabular}
			\quad
			$\Rightarrow$
			\quad
			\begin{tabular}{|c|c|c|c|c|}
				\hline
				& T & T & T & F \\
				\hline
				\hline
				$\sigma_3$ & 1 & 1 & -1 & -1 \\
				\hline
				$\sigma_4$ & -1 & 1 & -1 & 1 \\
				\hline
				$m=\sigma_3-\sigma_4$ & 2 & 0 & 0 & -2 \\
				\hline
			\end{tabular} 
			
		\end{center}
%		\caption{A way to map the 2-SAT clauses to the Ising variables.}
	\end{table} 
	
\end{frame}
%------------------------------------------------------------------------------------------

%------------------------------------------------------------------------------------------
\begin{frame}
	\frametitle{The 2-Satisfiability Problem}
%	\framesubtitle{Mapping a givent problem to the Hamiltonian}
	
	To encode the solution of the 2-SAT problem, the Hamiltonian can be designed for the first clause as 
	\begin{center}
		\begin{tabular}{|c|c|c|c|c|}
			
			\hline
			& T & T & T & F \\
			\hline
			\hline
			%		$\sigma_1$ & 1 & 1 & -1 & -1 \\
			%		\hline
			%		$\sigma_2$ & 1 & -1 & 1 & -1 \\
			%		\hline
			$m=\sigma_1+\sigma_2$ & 2 & 0 & 0 & -2 \\
			\hline
		\end{tabular} 
	\end{center}
	\begin{equation*}
	\begin{split}
	H & = m \cdot (m-2)\\
	& = \sigma_1^2 + \sigma_2^2 + 2\sigma_1 \sigma_2 -2\sigma_1 -2\sigma_2\\
	& = 2\sigma_1 \sigma_2 -2\sigma_1 -2\sigma_2 + const.\\
	\end{split}
	\end{equation*}
	Then one can correspond $h_1^z$, $h_2^z$, and $J_{12}$ in $H_{problem}$ to 2, 2, and -2 respectively, and repeat this procedure for all clauses. 
	\begin{equation*}
	H_{problem}= -\sum_{i,j}^N J_{ij}^z \sigma^z_i \sigma^z_j - \sum_{i}^N h_i^z \sigma^z_i
	\end{equation*} 
	%After this mapping is complete, the original 2-SAT problem can be solved with quantum annealing technique, because the ground state of the Ising Hamiltonian can be transformed back to the solution of the 2-SAT problem. \\  
	
	
\end{frame}
%------------------------------------------------------------------------------------------

\note{%	In order to encode the solution of the 2-SAT problem into the ground state of the Ising Hamiltonian, an Ising Hamiltonian can be built up with a magnetisation $m$. The Hamiltonian with magnetisation $m$ is given by 
	
	After comparing with the Hamiltonian mentioned in previous section, one can correspond $h_i^z$, $h_j^z$, and $J_{ij}$ to 2, 2, and -2 respectively. Then one just repeats this procedure for all clauses. 
	After this mapping is complete, the original 2-SAT problem can be solved with quantum annealing technique, because the ground state of the Ising Hamiltonian can be transformed back to the solution of the 2-SAT problem. \\  
	}

%------------------------------------------------------------------------------------------
\begin{frame}
	\frametitle{The Quantum Annealing Algorithm}
	\begin{tcolorbox}[title=Quantum Annealing Algorithm]
		Set up $H_{problem}$ according to a given problem. Combine $H_{init}$ and $H_{problem}$ to build $H(t)$. \\
		%\begin{equation*}
		%	H_{problem}=\sum_{i,j} \{J^x \sigma^x_i \sigma^x_j+J^y \sigma^y_i \sigma^y_j+J^z \sigma^z_i \sigma^z_j \}+ \sum_{i} \{h^x \sigma^x_i+h^y \sigma^y_i+h^z \sigma^z_i\}
		%\end{equation*}   
		
		1. Initialise the system to the ground state of $H_{init}$.\\ %The ground state of $H_{iniy}$ is designed in a way that it is simple to construct.\\
		%\begin{equation*}
		%	H_i=\sum_{i=1}^{N}h_i^x \sigma_i^x
		%\end{equation*}   
		
		
		
		2. Evolve the system by computing time-dependent Schrödinger equation for time $t$ according to a given time scheme. \\
		%\begin{equation*}
		%	T:linear 
		%\end{equation*}
		
		3. The system will end up in the ground state of the $H_{problem}$, if the total annealing T is long enough.\\
		
	\end{tcolorbox} 
\end{frame}
%------------------------------------------------------------------------------------------


\section{Simulation of a Quantum Annealer At Zero Temperature}

%------------------------------------------------------------------------------------------
\begin{frame}
	\frametitle{Time-Dependent Schrödinger Equation}
	
	\begin{itemize}
		\item The quantum annealing process can be described by time-dependent Schrödinger equation with the Hamiltonian given as
		\begin{equation*}
		H(t)=(1-\frac{t}{T} )H_{init}+(\frac{t}{T})H_{problem}
		\end{equation*}
		\item The solution of this is 
		\begin{equation*}
		\begin{split}
		\Psi(t+\tau)&=U(t+\tau,t)\Psi(t)\\
		&=\exp(-i\int_{t}^{t+\tau}H(\frac{t+\tau}{2})d\tau)\Psi(t),
		\end{split}
		\end{equation*}
		where $\tau$ is the time step, and $U(t+\tau)$ is a unitary matrix that transform system from $t$ to $t+\tau$.
		\item 	The time step should be small enough to keep $H(t)$ piecewise constant, and the solution can be rewritten as 
		\begin{equation*}
		\Psi(t+\tau) =\exp(-iH(\frac{t+\tau}{2})\tau)\Psi(t).
		\end{equation*}
		
	\end{itemize}
	


%   In order
%	to conduct numerical computation and ensure the unitary of the evolution operator, time is discretised to
%	small time step τ and H(τ) is assumed to be piecewise constant within these time steps. Therefore, the size
%	of the time step should be small enough to keep H(τ) piecewise. On the other hand, if the size is too small,
%	the computational time will be too long. A proper length should be tested out before going into further
%	simulation.
\end{frame}
%------------------------------------------------------------------------------------------

\note{	Namely, by solving time-dependent Schrödinger equation one can determine the final state of the system.\\ }


%%------------------------------------------------------------------------------------------
%\begin{frame}
%	\frametitle{Computational Basis}
%	\begin{itemize}
%		\item $\Psi$ is defined as a direct product state of the N single spin states in z-basis.
%		
%		\item Each single spin has two possible states, $\ket{\uparrow}$ or $\ket{\downarrow}$, which ,for convenience, are corresponded to $\ket{0}$ or $\ket{1}$. The relation is defined as 	
%		\begin{equation*}
%		\ket{\uparrow}\equiv \ket{0}, \ket{\downarrow} \equiv \ket{1}
%		\end{equation*}
%		
%		\item 
%		Thus, the single spin state can be written as a linear superposition of theses two basis state, 	
%		\begin{equation*}
%		\ket{\psi} = a(0)\ket{0} + a(1)\ket{1},
%		\end{equation*}
%		where $a(0)$ and $a(1)$ are the coefficient of the amplitude of each state.
%		
%		\item Furthermore, a N-spin system is made up of $2^N$ state vectors and it can be written as 	
%		\begin{equation*}
%		\label{psi_in_compuationalbasis}
%		\ket{\Psi} = a(00\dots0)\ket{00\dots0}+a(00\dots1)\ket{00\dots1}\cdots +a(11\dots1)\ket{11\dots1}+a(11\dots1)\ket{11\dots1},
%		\end{equation*}
%%		where $a(00\dots0), a(00\dots1)\dots, a(11\dots1)$ are the coefficients of each components. 
%		
%	\end{itemize}
%
%\end{frame}
%%------------------------------------------------------------------------------------------
%\note{With the basis denotes in 0 and 1 is convenient, because we can now consider tha basis state as a binary number notation from 0 to $2^N-1$. It will be easier denote the ground state as a binary number. }
%	\begin{equation*}
%	\ket{\uparrow}=\ket{0}=\left(\begin{array}{c} 1\\0 \end{array}\right), \ket{\downarrow}=\ket{1}=\left(\begin{array}{c} 0\\1 \end{array}\right)
%	\end{equation*}



%%------------------------------------------------------------------------------------------
%\begin{frame}
%	\frametitle{Hamiltonian for a System at Zero Temperature}
%	(I should remove this page.)
%\end{frame}
%%------------------------------------------------------------------------------------------

%%------------------------------------------------------------------------------------------
%\begin{frame}
%	\frametitle{The Full Diagonalisation method}
%		\begin{equation*}
%		\Psi(t+\tau) =\exp(-iH(\frac{t+\tau}{2})\tau)\Psi(t).
%		\end{equation*}
%	\begin{itemize}
%%		\item Hamiltonian H is an Hermitian operator formed by a $2^N \times 2^N$ matrix.
%%		\item the Hamiltonian $H$ can be transformed into a diagonal matrix $\Lambda$ spanned by its eigenvalues and the unitary matrix $V$ of its eigenvectors.
%		\item By diagonalisation, the time evolution can be calculated as 
%		\begin{equation*}
%		U(\tau) = exp(-i\tau H) = V exp(-i\tau \Lambda) V^\dagger.
%		\end{equation*}
%		\item The main limitation comes from the size of the quantum system.
%		\item This approach serves mostly as a tool to validate the correctness of other algorithm when solving a time-dependent Schrödinger equation.
%	\end{itemize}
%%	Hamiltonian H is an Hermitian operator formed by a $2^N \times 2^N$ matrix it has a complete set of eigenvectors and eigenvalues. This implies that the matrix H can be transformed into a diagonal matrix $\Lambda$ spanned by its eigenvalues and the unitary matrix $V$ of its eigenvectors. The transformation is given by $H = V\Lambda V^\dagger$. Thus the time evolution can be calculated as $U(\tau) = exp(-i\tau H) = V exp(-i\tau \Lambda) V^\dagger$. If $V$ and $\Lambda$ are obtained by diagonalising H, time evolution becomes a series of simple matrix multiplication. Since diagonalisation is a typical matrix problem, there are already standard software library can be made use of. \\
%%	
%\end{frame}
%%------------------------------------------------------------------------------------------
%
%\note{ Hamiltonian H is an Hermitian operator formed by a $2^N \times 2^N$ matrix.\\
%	   The complexity of the standard diagnolisation is in the order of $D^3$ D to the power of 3
%			The Hamiltonian $H$ can be transformed into a diagonal matrix $\Lambda$ spanned by its eigenvalues and the unitary matrix $V$ of its eigenvectors.}

%------------------------------------------------------------------------------------------
%\begin{frame}
%	\frametitle{The Suzuki-Trotter Product Formula Approach}
%	\begin{itemize}
%		\item The Suzuki-Trotter product formula approach can handle a larger quantum system than the full diagonalisation method does.
%		\item Suzuki-Trotter product formula approximate a unitary matrix exponentials by decomposing the matrix properly, that is 
%		\begin{equation*}
%		\label{suzuki-trotter product formula}
%		\begin{split}
%		U(t) &= \exp(-it H)\\
%		&=\exp(-it(H_1+H_2+\dots+H_K))\\
%		&=\lim_{m \to \infty}(\prod_{k=1}^{K}\exp(-it H_k/m))^m
%		\end{split}
%		\end{equation*}
%		\item  The crucial step of this approach is to choose $H_k$ properly in a way that it is easy enough to calculate there matrix exponential efficiently.
%	\end{itemize}
%\end{frame}
%%------------------------------------------------------------------------------------------


%------------------------------------------------------------------------------------------
\begin{frame}
	\frametitle{The Suzuki-Trotter Product Formula Approach}
	\begin{equation*}
		\begin{split}
		&H(t)=(1-\frac{t}{T} )H_{init}+(\frac{t}{T})H_{problem} \\
		&H_{init}= -\sum_{i=1}^{N}h_i^x \sigma_i^x ~,~~ H_{problem}= -\sum_{i}^N h_i^z \sigma^z_i - \sum_{i,j}^N J_{ij}^z \sigma^z_i \sigma^z_j\\
		\end{split}
	\end{equation*} 
	Based on the Hamiltonian used above, one way to construct the approximation is 
	\begin{equation*}
	\label{tilde_U}
	\begin{split}
	&\Psi(t+\tau) =\tilde{U}(t)\Psi(t)\\
	&\tilde{U}(t) = \exp(\frac{-i\tau H_{\sigma^{x,z}}}{2})\exp(-i\tau H_{\sigma_z \sigma_z})\exp(\frac{-i\tau H_{\sigma^{x,z}}}{2}).
	\end{split}
	\end{equation*}

\end{frame}
%------------------------------------------------------------------------------------------
%
%\begin{equation*}
%\label{tilde_U}
%\begin{split}
%\tilde{U}(\tau) = \exp(\frac{-i\tau H_{\sigma_\alpha}}{2})\exp(\frac{-i\tau H_{\sigma_x \sigma_x}}{2})\exp(\frac{-i\tau H_{\sigma_y \sigma_y}}{2})\exp(-i\tau H_{\sigma_z \sigma_z})\\\exp(\frac{-i\tau H_{\sigma_y \sigma_y}}{2})\exp(\frac{-i\tau H_{\sigma_x \sigma_x}}{2})\exp(\frac{-i\tau H_{\sigma_\alpha}}{2})
%\end{split}
%\end{equation*}
\note{The reason for decomposing Hamiltonian like this is that there is an analytical expression for the matrix expnential. The single spin term reduce to a 2x2 matrix multiplication. And the double spin term reduces to a 4x4 matrix multiplication.\\
	What is the complexity?}

%%------------------------------------------------------------------------------------------
%\begin{frame}
%	\frametitle{Validate Suzuki-Trotter Product Formula Approach with Full Diaganolisation Method}
%	\begin{figure}[h]
%		\centering
%		\includegraphics[height=150pt]{result_picture/compare_full_product/result_1e0.eps}
%		\caption{Insert caption here.}
%		
%	\end{figure}
%\end{frame}
%
%%------------------------------------------------------------------------------------------
%
%
%%------------------------------------------------------------------------------------------
%\begin{frame}
%	\frametitle{Validate Suzuki-Trotter Product Formula Approach with Full Diaganolisation Method}
%	\begin{figure}[h]
%		\centering
%		\includegraphics[height=150pt]{result_picture/compare_full_product/result_1e-1.eps}
%		\caption{Insert caption here.}
%		
%	\end{figure}
%\end{frame}
%
%%------------------------------------------------------------------------------------------

%------------------------------------------------------------------------------------------
\begin{frame}
	\frametitle{Simulation Result: Ground State Energy}
%	Annealing time = 1000
	\begin{figure}
		\centering
		\includegraphics[height=150pt]{result_picture/energy_evolution_product_formula/with_product_formula/Figure_productf_EnergyvsLambda.eps}
		\caption{An result with total annealing time = 1000}
	\end{figure}
\end{frame}
%------------------------------------------------------------------------------------------

%------------------------------------------------------------------------------------------
\begin{frame}
	\frametitle{Simulation Result: Success Probability}
	Annealing time = 1000
	\begin{figure}
		\centering
		\includegraphics[height=150pt]{result_picture/compare_full_product/Figure_result_1e-1.eps}
		\caption{An result with total annealing time = 1000}
	\end{figure}
\end{frame}
%------------------------------------------------------------------------------------------

%------------------------------------------------------------------------------------------
\begin{frame}
	\frametitle{Simulation Result: Spin Value}
		\begin{equation*}
		\begin{split}
		H_{init}= -\sum_{i=1}^{N}h_i^x \sigma_i^x ~,~~ H_{problem}= -\sum_{i}^N h_i^z \sigma^z_i  -\sum_{i,j}^N J_{ij}^z \sigma^z_i \sigma^z_j
		\end{split}
		\end{equation*} 
	\begin{figure}
		\centering
		\includegraphics[width=270pt]{result_picture/probability_spin_system/Figure_Annealing_sigma_value.eps}
%		\caption{An result with total annealing time = 1000}
		% \caption{Insert Caption here}
	\end{figure}
\end{frame}
%------------------------------------------------------------------------------------------

%------------------------------------------------------------------------------------------
\begin{frame}
	\frametitle{Effect of Total Annealing Time}
	\begin{figure}
		\centering
		\includegraphics[height=200pt]{result_picture/probability_spin_system/Figure_Annealing_probability.eps}
		
		% \caption{Insert Caption here}
	\end{figure}
\end{frame}
%------------------------------------------------------------------------------------------

%%------------------------------------------------------------------------------------------
%\begin{frame}
%	\frametitle{The Energy Spectrum and the Effect of Minimum Gap}
%	\begin{figure}
%		\centering
%		\includegraphics[height=200pt]{result_picture/energy_evolution_product_formula/with_product_formula/Figure_energy_spectrum.png}
%		
%		% \caption{Insert Caption here}
%	\end{figure}
%\end{frame}
%%------------------------------------------------------------------------------------------

%------------------------------------------------------------------------------------------
\begin{frame}
	\frametitle{The Energy Spectrum and the Minimum Gap}
	\begin{figure}
		\centering
		\includegraphics[height=200pt]{result_picture/energy_evolution_product_formula/with_product_formula/Figure_closelook_energy_spectrum.eps}
		
		% \caption{Insert Caption here}
	\end{figure}
\end{frame}
%------------------------------------------------------------------------------------------

%------------------------------------------------------------------------------------------
\begin{frame}
	\frametitle{Effect of Minimum Gap}
		\begin{equation*}
		\begin{split}
		&P_{adiabatic}(T) = 1-P_{diabatic}(T)\\
		&P_{diabatic}(T) = \exp(-c\cdot \Delta_{min}^2).
		\end{split}
		\end{equation*}
	\begin{figure}
		\centering
		\includegraphics[height=150pt]{result_picture/gap_result_environment/Figure_PvsGap_no_coupling_Temp1.eps}
		
		 \caption{A plot with 100 different 2-SAT problems.}
	\end{figure}
\end{frame}
%------------------------------------------------------------------------------------------




\section{Simulation of a Quantum Annealer At Finite Temperature}

%------------------------------------------------------------------------------------------
\begin{frame}
	\frametitle{Hamiltonian for a System at Finite Temperature}
	In order to simulate a system at finite temperature, the quantum system S is coupled to a heat bath B and the Hamiltonian of the entire system (i.e. S+B) is defined as
	
	\begin{equation*}
	H = H_S + H_B + gH_{SB},
	\end{equation*} 
	
	where $H_S$ is the Hamiltonian of the system, $H_B$ is the Hamiltonian of the heat bath, and $H_{SB}$ is the interaction between the subsystem S and the heat bath B with $g$ indicates the global coupling strength between S and B. 
\end{frame}
%------------------------------------------------------------------------------------------

%%------------------------------------------------------------------------------------------
%\begin{frame}
%	\frametitle{Canonical Ensemble}
%	An observable A can be calculated by going through all energy eigenstate of the heat bath as following:\\
%%		with 
%	
%	\begin{equation*}
%	\begin{split}
%	\label{psi_assemble_method}
%	\ket{\Psi(t)}&=e^{-iHt} \ket{E_B}\ket{S'}\\
%%	&=e^{-iHt}\sum_{i} c_i \frac{e^{\frac{-\beta E_B}{2}}}{\sqrt{Z}}\ket{\Psi_E}\ket{S'}
%	\end{split}
%	\end{equation*}
%	\begin{equation*}
%	\langle A(t)\rangle = \mathbf{Tr} A(t) \rho_B \rho_S= \sum_{B}\frac{e^{(\beta E_B)}}{Z}\bra{\Psi(t)}A\ket{\Psi(t)}
%	\end{equation*}
%	
%\end{frame}
%%------------------------------------------------------------------------------------------
%
%%------------------------------------------------------------------------------------------
%\begin{frame}
%	\frametitle{The Random Sampling Method}
%	Based on the hypothesis that with random sampling one can approximate the solution of a time-dependent Schrödinger Equation by solving a sample of randomly chosen initial state.
%	\begin{equation*}
%	\begin{split}
%	\mathbf{Tr}A=\sum_{n=1}^{D} \bra{\Psi_n}A\ket{\Psi_n}\\
%	\end{split}
%	\end{equation*}
%	
%	Then a random vector $\ket{\phi}$ can be constructed by choosing D complex random numbers with which mean is 0.
%	
%	\begin{equation*}
%	\begin{split}
%	\ket{\phi}= \sum_{n=1}^{D} c_n \ket{{\Psi_n}}, ~~ with ~~c_n \equiv f_n + i g_n\\
%	\end{split}
%	\end{equation*}
%\end{frame}
%%------------------------------------------------------------------------------------------
%
%
%%------------------------------------------------------------------------------------------
%\begin{frame}
%	\frametitle{The Random Sampling Method}
%	It follows that
%	
%	\begin{equation*}
%	\begin{split}
%	\bra{\phi}A\ket{\phi} = \sum_{m,n=1}^{D} c_m^\star c_n  \bra{\Psi_m}A\ket{\Psi_n}\\
%	\end{split}
%	\end{equation*}
%	
%	
%	It is possible to increase the accuracy by generate several samplings. If S realisations are sampled and then averaged out, it yields 
%	
%	\begin{equation*}
%	\begin{split}
%	\label{random_sampling}
%	\frac{1}{S}\sum_{p=1}^{S} \bra{\phi_p}A\ket{\phi_p} = \frac{1}{S} \sum_{p=1}^{S} \sum_{m,n=1}^{D} c_{m,p}^\star c_{n,p}  \bra{\Psi_{m,p}}A\ket{\Psi_{n,p}}
%	\end{split}
%	\end{equation*}
%	
%
%\end{frame}
%%------------------------------------------------------------------------------------------
%
%%------------------------------------------------------------------------------------------
%\begin{frame}
%	\frametitle{The Random Sampling Method}
%	If there is no correlation between the random numbers in different realisation, and the random number $f_n$ and $g_n$ are drawn from an even and symmetric probability distribution, the argument can further be made as
%	\begin{equation*}
%	\begin{split}
%	\lim_{S \to \infty} \frac{1}{S} \sum_{p=1}^{S}  \bra{\phi_p}A\ket{\phi_p} &= E(|c|^2)\sum_{n=1}^{D}\bra{\Psi_n}A\ket{\Psi_n}\\
%	&=E(|c|^2) TrA,
%	\end{split}
%	\end{equation*}
%	where $E(\cdot)$ is the expectation value based on the probability distribution used to draw $c_n$.
%	
%
%	
%	
%\end{frame}
%%------------------------------------------------------------------------------------------






%------------------------------------------------------------------------------------------
\begin{frame}
	\frametitle{Effect of Total Annealing Time at different Temperature}

%	Temperature = 0.02, 1, 1000
	\begin{figure}
		\centering
		\includegraphics[width=0.320\textwidth]{result_picture/JvsT_plot/Exact/Figure_JvsT_T2e-2.eps}
		\hfill
		\includegraphics[width=0.320\textwidth]{result_picture/JvsT_plot/Exact/Figure_JvsT_T1e0.eps}
		\hfill
		\includegraphics[width=0.320\textwidth]{result_picture/JvsT_plot/Exact/Figure_JvsT_T1e3.eps}
		 \caption{Left: Temp. = 0.02; Middle: Temp. = 1; Right: Temp. = 1000}
	\end{figure}
\end{frame}
%------------------------------------------------------------------------------------------

%------------------------------------------------------------------------------------------
\begin{frame}
	\frametitle{Effect of Total Annealing Time at different Temperature}
	
%	Coherent - Non-equilibrium - Quasistatic
	\begin{figure}
		\centering
		\includegraphics[height=150pt]{result_picture/JvsT_plot/Exact/Figure_JvsT_T1e0.eps}
		 \caption{A result at Temperature = 1}
	\end{figure}
\end{frame}
%------------------------------------------------------------------------------------------

%------------------------------------------------------------------------------------------
\begin{frame}
	\frametitle{Effect of Total Annealing Time at different Temperature}
	Temperature = 1\\
	Coherent - Non-equilibrium - Quasistatic
	\begin{figure}
		\centering
		\includegraphics[width=0.475\textwidth]{result_picture/JvsT_plot/Exact/Figure_JvsT_T1e0_coherent.eps}
		\hfill
		\includegraphics[width=0.475\textwidth]{Coherent.png}\footnotemark[1]
		% \caption{Insert Caption here}
	\end{figure}
	
	\footnote[1]{Amin, M. H. (2015). Searching for quantum speedup in quasistatic quantum annealers. Physical Review A - Atomic, Molecular, and Optical Physics, 92(5), 1–5. https://doi.org/10.1103/PhysRevA.92.052323}
\end{frame}
%------------------------------------------------------------------------------------------

%------------------------------------------------------------------------------------------
\begin{frame}
	\frametitle{Effect of Total Annealing Time at different Temperature}
	Temperature = 1000\\
	Coherent - Non-equilibrium -X- Quasistatic
	\begin{figure}
		\centering
		\includegraphics[width=0.475\textwidth]{result_picture/JvsT_plot/Exact/Figure_JvsT_T1e3_coherent.eps}
		\hfill
		\includegraphics[width=0.475\textwidth]{Coherent.png}\footnotemark[1]		
		% \caption{Insert Caption here}
	\end{figure}
	\footnote[1]{Amin, M. H. (2015). Searching for quantum speedup in quasistatic quantum annealers. Physical Review A - Atomic, Molecular, and Optical Physics, 92(5), 1–5. https://doi.org/10.1103/PhysRevA.92.052323}
\end{frame}
%------------------------------------------------------------------------------------------

%------------------------------------------------------------------------------------------
\begin{frame}
	\frametitle{Effect of Minimum Gap under different Coupling Strength }

	\begin{figure}
		\centering
		\includegraphics[height=220pt]{result_picture/gap_result_avoid_duplicate/gap_result_environment/avoided_duplicate/Figure_PvsGap_T2e2_Temp1.eps}
		
		% \caption{Insert Caption here}
	\end{figure}
\end{frame}
%------------------------------------------------------------------------------------------

%------------------------------------------------------------------------------------------
\begin{frame}
	\frametitle{Effect of Minimum Gap under different Temperature }
	
	\begin{figure}
		\centering
		\includegraphics[height=220pt]{result_picture/gap_result_avoid_duplicate/gap_result_temperatur_G02_Time1e3/avoided_duplicate/Figure2_PvsGap_T1e3_G02.eps}	
		% \caption{Insert Caption here}
	\end{figure}
%	/result_picture/gap_result_avoid_duplicate/gap_result_temperatur_G02_Time1e3
\end{frame}
%------------------------------------------------------------------------------------------

%------------------------------------------------------------------------------------------
\begin{frame}
	\frametitle{Conclusion}
	\begin{itemize}
		\item The quantum annealing does provide a different approach to solve 2-SAT problem.
		\item The quantum annealing process can be influenced not only by the total annealing time, minimum gap, but also the temperature.
		\item A quantum annealing process at finite temperature will go through a coherent state, non-equlibrium state, and finally a quasistatic state as mentioned in a previous paper.
		\item The simulation done in this study can be tested on an D-Wave annealer.
	\end{itemize}
\end{frame}
%------------------------------------------------------------------------------------------

%%%%%%%%%%%
%%%%%%%%%%%
%%%%%%%%%%%
\appendix

%------------------------------------------------------------------------------------------
\mode<presentation>{
	{   
		\setbeamercolor{background canvas}{bg=black}
		\begin{frame}[plain]{}
		\end{frame}
	}
}
\mode<presentation>{\setbeamercolor{background canvas}{bg=white}}
\mode*
%------------------------------------------------------------------------------------------

%------------------------------------------------------------------------------------------
\begin{frame}
	\centering
	Appendix
\end{frame}
%------------------------------------------------------------------------------------------


%------------------------------------------------------------------------------------------
\begin{frame}
	\frametitle{Appendix: Computational Basis}
	\begin{itemize}
		\item $\Psi$ is defined as a direct product state of the N single spin states in z-basis.
		
		\item Each single spin has two possible states, $\ket{\uparrow}$ or $\ket{\downarrow}$, which ,for convenience, are corresponded to $\ket{0}$ or $\ket{1}$. The relation is defined as 	
		\begin{equation*}
		\ket{\uparrow}\equiv \ket{0}, \ket{\downarrow} \equiv \ket{1}
		\end{equation*}
		
		\item 
		Thus, the single spin state can be written as a linear superposition of theses two basis state, 	
		\begin{equation*}
		\ket{\psi} = a(0)\ket{0} + a(1)\ket{1},
		\end{equation*}
		where $a(0)$ and $a(1)$ are the coefficient of the amplitude of each state.
		
		\item Furthermore, a N-spin system is made up of $2^N$ state vectors and it can be written as 	
		\begin{equation*}
		\label{psi_in_compuationalbasis}
		\ket{\Psi} = a(00\dots0)\ket{00\dots0}+a(00\dots1)\ket{00\dots1}\cdots +a(11\dots1)\ket{11\dots1}+a(11\dots1)\ket{11\dots1},
		\end{equation*}
		%		where $a(00\dots0), a(00\dots1)\dots, a(11\dots1)$ are the coefficients of each components. 
		
	\end{itemize}
	
\end{frame}
%------------------------------------------------------------------------------------------
\note{With the basis denotes in 0 and 1 is convenient, because we can now consider tha basis state as a binary number notation from 0 to $2^N-1$. It will be easier denote the ground state as a binary number. }

%------------------------------------------------------------------------------------------
\begin{frame}
	\frametitle{Appendix: The Full Diagonalisation method}
	\begin{equation*}
	\Psi(t+\tau) =\exp(-iH(\frac{t+\tau}{2})\tau)\Psi(t).
	\end{equation*}
	\begin{itemize}
		%		\item Hamiltonian H is an Hermitian operator formed by a $2^N \times 2^N$ matrix.
		%		\item the Hamiltonian $H$ can be transformed into a diagonal matrix $\Lambda$ spanned by its eigenvalues and the unitary matrix $V$ of its eigenvectors.
		\item By diagonalisation, the time evolution can be calculated as 
		\begin{equation*}
		U(\tau) = exp(-i\tau H) = V exp(-i\tau \Lambda) V^\dagger.
		\end{equation*}
		\item The main limitation comes from the size of the quantum system.
		\item This approach serves mostly as a tool to validate the correctness of other algorithm when solving a time-dependent Schrödinger equation.
	\end{itemize}
	%	Hamiltonian H is an Hermitian operator formed by a $2^N \times 2^N$ matrix it has a complete set of eigenvectors and eigenvalues. This implies that the matrix H can be transformed into a diagonal matrix $\Lambda$ spanned by its eigenvalues and the unitary matrix $V$ of its eigenvectors. The transformation is given by $H = V\Lambda V^\dagger$. Thus the time evolution can be calculated as $U(\tau) = exp(-i\tau H) = V exp(-i\tau \Lambda) V^\dagger$. If $V$ and $\Lambda$ are obtained by diagonalising H, time evolution becomes a series of simple matrix multiplication. Since diagonalisation is a typical matrix problem, there are already standard software library can be made use of. \\
	%	
\end{frame}
%------------------------------------------------------------------------------------------

\note{ Hamiltonian H is an Hermitian operator formed by a $2^N \times 2^N$ matrix.\\
	The complexity of the standard diagnolisation is in the order of $D^3$ D to the power of 3
	The Hamiltonian $H$ can be transformed into a diagonal matrix $\Lambda$ spanned by its eigenvalues and the unitary matrix $V$ of its eigenvectors.}


%------------------------------------------------------------------------------------------
\begin{frame}
	\frametitle{Appendix: The Suzuki-Trotter Product Formula Approach}
	\begin{itemize}
		\item The Suzuki-Trotter product formula approach can handle a larger quantum system than the full diagonalisation method does.
		\item Suzuki-Trotter product formula approximate a unitary matrix exponentials by decomposing the matrix properly, that is 
		\begin{equation*}
		\label{suzuki-trotter product formula}
		\begin{split}
		U(t) &= \exp(-it H)\\
		&=\exp(-it(H_1+H_2+\dots+H_K))\\
		&=\lim_{m \to \infty}(\prod_{k=1}^{K}\exp(-it H_k/m))^m
		\end{split}
		\end{equation*}
		\item  The crucial step of this approach is to choose $H_k$ properly in a way that it is easy enough to calculate there matrix exponential efficiently.
	\end{itemize}
\end{frame}
%------------------------------------------------------------------------------------------

%------------------------------------------------------------------------------------------
\begin{frame}
	\frametitle{Appendix: The Suzuki-Trotter Product Formula Approach and The Full Diaganolisation Method}
	\begin{figure}[h]
		\centering
		\includegraphics[height=150pt]{result_picture/compare_full_product/result_1e0.eps}
%		\caption{Insert caption here.}
		
	\end{figure}
\end{frame}

%------------------------------------------------------------------------------------------


%------------------------------------------------------------------------------------------
\begin{frame}
	\frametitle{Appendix: The Suzuki-Trotter Product Formula Approach and the Full Diaganolisation Method}
	\begin{figure}[h]
		\centering
		\includegraphics[height=150pt]{result_picture/compare_full_product/result_1e-1.eps}
%		\caption{Insert caption here.}
		
	\end{figure}
\end{frame}

%------------------------------------------------------------------------------------------


%------------------------------------------------------------------------------------------
\begin{frame}
	\frametitle{Appendix: The Energy Spectrum and the Effect of Minimum Gap}
	\begin{figure}
		\centering
		\includegraphics[height=200pt]{result_picture/energy_evolution_product_formula/with_product_formula/Figure_energy_spectrum.png}
		
		% \caption{Insert Caption here}
	\end{figure}
\end{frame}
%------------------------------------------------------------------------------------------

%------------------------------------------------------------------------------------------
\begin{frame}
	\frametitle{Appendix: Canonical Ensemble}
	An observable A can be calculated by going through all energy eigenstate of the heat bath as following:\\
	%		with 
	
	\begin{equation*}
	\begin{split}
	\label{psi_assemble_method}
	\ket{\Psi(t)}&=e^{-iHt} \ket{E_B}\ket{S'}\\
	%	&=e^{-iHt}\sum_{i} c_i \frac{e^{\frac{-\beta E_B}{2}}}{\sqrt{Z}}\ket{\Psi_E}\ket{S'}
	\end{split}
	\end{equation*}
	\begin{equation*}
	\langle A(t)\rangle = \mathbf{Tr} A(t) \rho_B \rho_S= \sum_{B}\frac{e^{(\beta E_B)}}{Z}\bra{\Psi(t)}A\ket{\Psi(t)}
	\end{equation*}
	
\end{frame}
%------------------------------------------------------------------------------------------

%------------------------------------------------------------------------------------------
\begin{frame}
	\frametitle{Appendix: The Random Sampling Method}
	Based on the hypothesis that with random sampling one can approximate the solution of a time-dependent Schrödinger Equation by solving a sample of randomly chosen initial state.
	\begin{equation*}
	\begin{split}
	\mathbf{Tr}A=\sum_{n=1}^{D} \bra{\Psi_n}A\ket{\Psi_n}\\
	\end{split}
	\end{equation*}
	
	Then a random vector $\ket{\phi}$ can be constructed by choosing D complex random numbers with which mean is 0.
	
	\begin{equation*}
	\begin{split}
	\ket{\phi}= \sum_{n=1}^{D} c_n \ket{{\Psi_n}}, ~~ with ~~c_n \equiv f_n + i g_n\\
	\end{split}
	\end{equation*}
\end{frame}
%------------------------------------------------------------------------------------------


%------------------------------------------------------------------------------------------
\begin{frame}
	\frametitle{Appendix: The Random Sampling Method}
	It follows that
	
	\begin{equation*}
	\begin{split}
	\bra{\phi}A\ket{\phi} = \sum_{m,n=1}^{D} c_m^\star c_n  \bra{\Psi_m}A\ket{\Psi_n}\\
	\end{split}
	\end{equation*}
	
	
	It is possible to increase the accuracy by generate several samplings. If S realisations are sampled and then averaged out, it yields 
	
	\begin{equation*}
	\begin{split}
	\label{random_sampling}
	\frac{1}{S}\sum_{p=1}^{S} \bra{\phi_p}A\ket{\phi_p} = \frac{1}{S} \sum_{p=1}^{S} \sum_{m,n=1}^{D} c_{m,p}^\star c_{n,p}  \bra{\Psi_{m,p}}A\ket{\Psi_{n,p}}
	\end{split}
	\end{equation*}
	
	
\end{frame}
%------------------------------------------------------------------------------------------

%------------------------------------------------------------------------------------------
\begin{frame}
	\frametitle{Appendix: The Random Sampling Method}
	If there is no correlation between the random numbers in different realisation, and the random number $f_n$ and $g_n$ are drawn from an even and symmetric probability distribution, the argument can further be made as
	\begin{equation*}
	\begin{split}
	\lim_{S \to \infty} \frac{1}{S} \sum_{p=1}^{S}  \bra{\phi_p}A\ket{\phi_p} &= E(|c|^2)\sum_{n=1}^{D}\bra{\Psi_n}A\ket{\Psi_n}\\
	&=E(|c|^2) TrA,
	\end{split}
	\end{equation*}
	where $E(\cdot)$ is the expectation value based on the probability distribution used to draw $c_n$.
	
	
	
	
\end{frame}
%------------------------------------------------------------------------------------------


%------------------------------------------------------------------------------------------
\begin{frame}
	\frametitle{Appendix: Results of the Random Sampling Method}
	\begin{figure}
		\centering
		\includegraphics[width=0.495\textwidth]{result_picture/JvsT_plot/T1/Figure_JvsT_T1e0_run1.eps}
		\hfill
		\includegraphics[width=0.495\textwidth]{result_picture/JvsT_plot/Exact/Figure_JvsT_T1e0.eps}		
		
		% \caption{Insert Caption here}
	\end{figure}
\end{frame}
%------------------------------------------------------------------------------------------

%------------------------------------------------------------------------------------------
\begin{frame}
	\frametitle{Appendix: Results of the Random Sampling Method}
	\begin{figure}
		\centering
		\includegraphics[width=0.495\textwidth]{result_picture/JvsT_plot/T1/Figure_JvsT_T1e0_run5.eps}
		\hfill
		\includegraphics[width=0.495\textwidth]{result_picture/JvsT_plot/Exact/Figure_JvsT_T1e0.eps}		
		
		% \caption{Insert Caption here}
	\end{figure}
\end{frame}
%------------------------------------------------------------------------------------------


%------------------------------------------------------------------------------------------
\begin{frame}
	\frametitle{Appendix: Results of the Random Sampling Method}
	\begin{figure}
		\centering
		\includegraphics[width=0.495\textwidth]{result_picture/JvsT_plot/T1/Figure_JvsT_T1e0_run8.eps}
		\hfill
		\includegraphics[width=0.495\textwidth]{result_picture/JvsT_plot/Exact/Figure_JvsT_T1e0.eps}		
		
		% \caption{Insert Caption here}
	\end{figure}
\end{frame}
%------------------------------------------------------------------------------------------

%------------------------------------------------------------------------------------------
\begin{frame}
	\frametitle{Appendix: Results of the Random Sampling Method}
	\begin{figure}
		\centering
		\includegraphics[width=0.495\textwidth]{result_picture/JvsT_plot/T1/Figure_JvsT_T1e0_random.eps}
		\hfill
		\includegraphics[width=0.495\textwidth]{result_picture/JvsT_plot/Exact/Figure_JvsT_T1e0.eps}		
		
		% \caption{Insert Caption here}
	\end{figure}
\end{frame}
%------------------------------------------------------------------------------------------

%%%%%%%%%%%%%%
%%%%%%%%%%%%%%
%%%%%%%%%%%%%%
\end{document}

